\documentclass[12pt, reqno]{amsart}

%%%%%%%%%%%%%%%%%%%%%  MY STUFF %%%%%%%%%%%%%%%%%%%%%%%%%%%%%%%%%%
%\makeatletter
%\g@addto@macro{\endabstract}{\@setabstract}
%\makeatother



%\usepackage{epsfig}
\usepackage{graphics, stackrel}
\usepackage{amsmath, amssymb, amsthm}
\usepackage{graphicx}
\usepackage{verbatim}
\usepackage{amsfonts}
\usepackage{natbib}
\usepackage{enumitem}
%font
%\usepackage{lmodern}
\usepackage[T1]{fontenc}
\usepackage{mathpazo}
%\usepackage{tgpagella}

%subfloats / figures
\usepackage{caption}
\usepackage{subcaption}

% For pandas latex tables
\usepackage{booktabs}


\usepackage{fancyvrb}
\usepackage[usenames,dvipsnames,svgnames,table]{xcolor}
\usepackage{mdwlist}

\usepackage[citecolor=blue, colorlinks=true, linkcolor=blue]{hyperref}


% lists
\usepackage{enumitem}
\setlist[enumerate]{itemsep=2pt,topsep=3pt}
\setlist[itemize]{itemsep=2pt,topsep=3pt}
\setlist[enumerate,1]{label=(\alph*)}

\usepackage{mathrsfs}  % caligraphic
%\usepackage{stix} 
\usepackage{bbm}
\usepackage{bm}        % bold symbols


%% page layout
\usepackage[left=1.25in, right=1.25in, top=1.0in, bottom=1.15in, includehead, includefoot]{geometry}

% nice inequalities
\renewcommand{\leq}{\leqslant}
\renewcommand{\geq}{\geqslant}

% inner product
\providecommand{\inner}[1]{\left\langle{#1}\right\rangle}
\providecommand{\innerp}[1]{\left\langle{#1}\right\rangle_\pi}


\usepackage[ruled, linesnumbered]{algorithm2e}

%extra spacing
\renewcommand{\baselinestretch}{1.25}

%horizonal line
\newcommand{\HRule}{\rule{\linewidth}{0.3mm}}

% skip a line between paragraphs, no indentation
\setlength{\parskip}{1.5ex plus0.5ex minus0.5ex}
\setlength{\parindent}{0pt}

% footnote without a maker (blfootnote)
\newcommand\blfootnote[1]{%
  \begingroup
  \renewcommand\thefootnote{}\footnote{#1}%
  \addtocounter{footnote}{-1}%
  \endgroup
}

\DeclareMathOperator{\Span}{span}
\DeclareMathOperator{\diag}{diag}
\DeclareMathOperator*{\argmin}{arg\,min}
\DeclareMathOperator*{\argmax}{arg\,max}

\DeclareMathOperator{\cl}{cl}
%\DeclareMathOperator{\overset{\circ}}{int}
\DeclareMathOperator{\Prob}{Prob}
\DeclareMathOperator{\determinant}{det}
\DeclareMathOperator{\Var}{Var}
\DeclareMathOperator{\Cov}{Cov}
\DeclareMathOperator{\graph}{graph}

% mics short cuts and symbols
\newcommand{\st}{\ensuremath{\ \mathrm{s.t.}\ }}
\newcommand{\setntn}[2]{ \{ #1 : #2 \} }
\newcommand{\fore}{\therefore \quad}
\newcommand{\preqsd}{\preceq_{sd} }
\newcommand{\toas}{\stackrel {\textrm{ \scriptsize{a.s.} }} {\to} }
\newcommand{\tod}{\stackrel { d } {\to} }
\newcommand{\tou}{\stackrel { u } {\to} }
\newcommand{\toweak}{\stackrel { w } {\to} }
\newcommand{\topr}{\stackrel { p } {\to} }
\newcommand{\disteq}{\stackrel { \mathscr D } {=} }
\newcommand{\eqdist}{\stackrel {\textrm{ \scriptsize{d} }} {=} }
\newcommand{\iidsim}{\stackrel {\textrm{ {\sc iid }}} {\sim} }
\newcommand{\1}{\mathbbm 1}
\newcommand{\la}{\langle}
\newcommand{\ra}{\rangle}
\newcommand{\dee}{\,{\rm d}}
\newcommand{\og}{{\mathbbm G}}
\newcommand{\ctimes}{\! \times \!}
\newcommand{\sint}{{\textstyle\int}}

\newcommand{\given}{\, | \,}
\newcommand{\A}{\forall}

% d for integrals
\newcommand*\diff{\mathop{}\!\mathrm{d}}
\newcommand*\e{\mathrm{e}}


% Special symbols and shortcuts
\newcommand{\bmeta}{\bm{\eta}}
\newcommand{\bmxi}{\bm{\xi}}

\newcommand{\infot}{\fF_t}

\newcommand{\pspace}{\mathscr{P}(\mathsf{X})}
\newcommand{\cspace}{\mathscr{C}(\mathsf{X})}

%\renewcommand{\times}{\! \times \!}

\newcommand{\aA}{\mathscr A}
\newcommand{\cC}{\mathscr C}
\newcommand{\sS}{\mathcal S}
\newcommand{\bB}{\mathscr B}
\newcommand{\oO}{\mathcal O}
\newcommand{\gG}{\mathcal G}
\newcommand{\hH}{\mathcal H}
\newcommand{\kK}{\mathcal K}
\newcommand{\iI}{\mathcal I}
\newcommand{\eE}{\mathcal E}
\newcommand{\fF}{\mathscr F}
\newcommand{\qQ}{\mathcal Q}
\newcommand{\tT}{\mathcal T}
\newcommand{\xX}{\mathcal X}
\newcommand{\yY}{\mathcal Y}
\newcommand{\rR}{\mathcal R}
\newcommand{\zZ}{\mathcal Z}
\newcommand{\wW}{\mathcal W}
\newcommand{\uU}{\mathcal U}
\newcommand{\lL}{\mathcal L}

\newcommand{\mM}{\mathcal M}
\newcommand{\dD}{\mathcal D}
\newcommand{\pP}{\mathcal P}

\newcommand{\vV}{\mathcal V}

\newcommand{\Bsf}{\mathsf B}
\newcommand{\Hsf}{\mathsf H}
\newcommand{\Vsf}{\mathsf V}

\newcommand{\BB}{\mathbbm B}
\newcommand{\DD}{\mathbbm D}
\newcommand{\RR}{\mathbbm R}
\newcommand{\CC}{\mathbbm C}
\newcommand{\QQ}{\mathbbm Q}
\newcommand{\NN}{\mathbbm N}
\newcommand{\GG}{\mathbbm G}
\newcommand{\UU}{\mathbbm U}
\newcommand{\TT}{\mathbbm T}
\newcommand{\YY}{\mathbbm Y}
\newcommand{\ZZ}{\mathbbm Z}
\newcommand{\HH}{\mathbbm H}
\newcommand{\KK}{\mathbbm K}
\newcommand{\MM}{\mathbbm M}
\newcommand{\PP}{\mathbbm P}
\newcommand{\EE}{\mathbbm E}


\newcommand{\bH}{\mathbf H}
\newcommand{\bT}{\mathbf T}

\newcommand{\var}{\mathbbm V}

\newcommand{\XX}{\mathsf X}
\newcommand{\II}{\mathsf I}
\newcommand{\WW}{\mathsf W}

\renewcommand{\phi}{\varphi}
\renewcommand{\epsilon}{\varepsilon}

\newcommand{\bP}{\mathbf P}
\newcommand{\bQ}{\mathbf Q}
\newcommand{\bE}{\mathbf E}
\newcommand{\bM}{\mathbf M}
\newcommand{\bX}{\mathbf X}
\newcommand{\bY}{\mathbf Y}

\theoremstyle{plain}
\newtheorem{theorem}{Theorem}[section]
\newtheorem{corollary}[theorem]{Corollary}
\newtheorem{lemma}[theorem]{Lemma}
\newtheorem{proposition}[theorem]{Proposition}


\theoremstyle{definition}
\newtheorem{definition}{Definition}[section]
\newtheorem{axiom}{Axiom}[section]
\newtheorem{example}{Example}[section]
\newtheorem{remark}{Remark}[section]
\newtheorem{notation}{Notation}[section]
\newtheorem{assumption}{Assumption}[section]
\newtheorem{condition}{Condition}[section]


%\DeclareTextFontCommand{\emph}{\bfseries}

%%%%%%%%%%%%%%%%%% end my preamble %%%%%%%%%%%%%%%%%%%%%%%%%%%%%%%%%%%






\begin{document}


\title{}




\begin{center}
    \LARGE 
    Solving High-Dimensional Asset Pricing Models via Newton--Kantorovich
    Iteration
    \blfootnote{To be added}
 
    \vspace{1em}

    \large
    Chase Coleman\textsuperscript{a}, Pablo Levi\textsuperscript{b}, John
    Stachurski\textsuperscript{c}, \\ Ole Wilms\textsuperscript{d} and Junnan Zhang\textsuperscript{e} \par \bigskip

    \small
    \textsuperscript{a} Research School of Economics, Australian National University \\ 
    \textsuperscript{b} Other affiliations to be added  \\ \bigskip

    \normalsize
    \today
\end{center}


\begin{abstract}
    To be written
    \vspace{1em}

    \noindent
    \textit{JEL Classifications:} D81, G11 \\
    \textit{Keywords:} Asset pricing, wealth-consumption ratio, automatic differentiation
\end{abstract}





\maketitle


%section
\section{Introduction}

As with many sub-fields in economics and finance, researchers working on asset
pricing face the need to handle a growing list of state variables when trying to
match existing theory to the data.  For example, while the seminal paper of
\cite{bansal2004risks} used just two state variables to model the
wealth-consumption ratio, subsequent work has added many features to the
consumption side of this model.  A representative study,
\cite{schorfheide2018identifying} adds preference shocks and an additional
stochastic volatility term, leading to four state variables.
\cite{gomez2021important} adds two more state variables to handle expectations
of inflation.  

To date, the standard method for handling these kinds of models has been
log-linearization.  However, the underlying models are highly nonlinear and,
moreover, it has been shown that these nonlinearities matter for endogenous
quantities of interest, such as the risk premium (see, e.g.,
\cite{pohl2018higher}).  Hence researchers are seeking better global solution
methods that capture the full impact of underlying nonlinearities.

At the same time, for a model of this type, even a moderate number of state
variables leads to challenging computational problems.  One reason is the
nonlinearities discussed above.  Another is that, in standard calibrations,
discount factors are very close to unity (e.g., 0.998 in
\cite{bansal2004risks}), implying slow convergence of some solution methods.
Moreover, the consumption problem is typically embedded in a larger model with
still more state variables, so fast and accurate solutions are essential.  

In this paper we show how to rapidly approximate global solutions to the kinds
of asset pricing models listed above using Newton--Kantorovich iteration backed
by three features: automatic differentiation, just-in-time compilation and 
execution on hardware accelerators such as GPUs. 

To implement these feastures we use the software library JAX. 

\textcolor{DarkOrchid}{Short history of JAX, discussion of its capabilities.}

\textcolor{DarkOrchid}{Some explanation of what we achieve. Review of related
literature.}

\textcolor{DarkOrchid}{Some benefits of our method. 
    \begin{itemize}
        \item We can potentially differentiate the fixed point with respect to the
            parameters. These gradients help us how the SDF and WC ratio respond
            to shifts in underlying parameters.  They can also provide Jacobians
            for gradient decent.
        \item Autodiff means it's possible to plug other fixed point operators, associated with
            other specifications of recursive utility, directly into the code.
            There's no need to compute gradients in each case.
        \item By switching from successive approximation to Newton's method, using
            autodifferentiation to compute the Jacobian, we change the problem
            from many small iterations to a small number of computationally
            expensive ones, which offer better opportunities for parallelization. 
        \item Autodifferentiation is important here, as compared to numerical
            derivatives, since convergence is fragile for this operator (very slow
            rate of convergence).
        \item JAX allows us to invert the Jacobian without actually instantiating
            the full Jacobian matrix.   This is a key feature for successfully
            solving high dimensional problems. 
    \end{itemize}
}


\section{Asset Pricing Background}

In this section we introduce the model and then progress to stating the
functional equation for the wealth-consumption ration.

In discrete-time no-arbitrage environments, the equilibrium price process
$\{P_t\}_{t \geq 0}$ associated with a cash flow $\{G_t\}_{t \geq 1}$ obeys
the fundamental recursion
%
\begin{equation}
    \label{eq:pd}
    P_t
    = \EE_t \, M_{t+1} ( P_{t+1} + G_{t+1} )
    \quad \text{for all } t \geq 0,
\end{equation}
%
where $\{ M_t\}$ is the sequence of single period stochastic discount factors
(see, for example,~\cite{kreps1981arbitrage}, \cite{hansen_richard:1987}
or~\cite{duffie2001dynamic}).  Most of the well-known ``puzzles'' in asset
pricing theory relate to the difficulty of matching \eqref{eq:pd} to the data
across a diverse range of asset classes. Attempts to resolve these puzzles
typically involve relatively sophisticated models for the stochastic discount
factor (SDF) process $\{ M_t\}$.

In the model, the wealth-consumption ratio obeys
%
\begin{equation*}
    \beta^{\theta}
    \EE_t
    \left[
    \left( \frac{\lambda_{t+1}}{\lambda_t} \right)^\theta
        \left( \frac{C_{t+1}}{C_t} \right)^{1-\gamma}
        \left( \frac{w(X_{t+1})}{w(X_t)-1} \right)^\theta
    \right] = 1,
\end{equation*}
%
where $\{ X_t \}_{t \geq 0}$ is a stationary time-homogeneous Markov process on
$\XX \subset \RR^d$.
Rearranging the previous expression gives
%
\begin{align*}
    (w(X_t)-1)^\theta
    & = \beta^{\theta}
    \EE_t
    \left[
    \left( \frac{\lambda_{t+1}}{\lambda_t} \right)^\theta
        \left( \frac{C_{t+1}}{C_t} \right)^{1-\gamma}
        w(X_{t+1})^\theta
    \right]
    \\
    & = \beta^{\theta}
    \EE_t
    \left[
        \exp
        \left\{ 
            \theta g_{\lambda, t+1} + (1-\gamma) g_{c, t+1}
        \right\}
        w(X_{t+1})^\theta
    \right],
\end{align*}
%
where
%
\begin{equation}
    \label{eq:kappa}
    g_{c, t+1}
    := \ln \frac{C_{t+1}}{C_t}
    \quad \text{and} \quad
    g_{\lambda, t+1}
    := \ln \frac{\lambda_{t+1}}{\lambda_t}.
\end{equation}
%

Let $\HH$ be the linear operator defined by
%
\begin{equation}\label{eq:defk}
    (\HH f)(x) = \EE_x 
        \, f(X_{t+1})  \,
        \exp
        \left\{ 
            \theta g_{\lambda, t+1} + (1-\gamma) g_{c, t+1}
        \right\}
\end{equation}
%
at each $x \in \XX$,  where $\EE_x$ conditions on $X_t = x$.  With this notation
we can now write the equation for the wealth-consumption ratio as
%
\begin{equation}
    (w(x)-1)^\theta
    = \beta^{\theta}
    \left[
        (\HH w)(x)^\theta
    \right].
    \label{eq:wcnt}
\end{equation}
%
Rearranging once more gives the functional equation
%
\begin{equation}
    w = 1 + \beta ( \HH w^\theta )^{1/\theta}.
    \label{eq:fe}
\end{equation}
%
A function $w$ solves this equation if and only if $w$ is a fixed point
of the operator $\TT$ defined by 
%
\begin{equation}\label{eq:wcop}
    \TT w = 1 + \beta \,  (\HH w^\theta)^{1/\theta}.
\end{equation}
%



\subsection{Example: The SSY Case}

To implement the Koopmans operator $\TT$, we need to specify the linear operator
$\HH$.  Here we specify $\HH$ for the 
the long run risk model of \cite{schorfheide2018identifying}.  

In this model, the state process takes the form 
%
\begin{equation}\label{eq:state}
    X_t := (h_{\lambda, t}, h_{c, t}, h_{z, t}, z_t)  
\end{equation}
%
where
\begin{equation*}
    z_{t+1} = \rho \, z_t + (1 - \rho^2)^{1/2} \, \sigma_{z, t} \, \epsilon_{t+1}
\end{equation*}
%
and
%
\begin{equation*}
    \sigma_{i,t} = \phi_i \, \bar{\sigma} \, \exp(h_{i,t}),
    \qquad
    h_{i, t+1} = \rho_i \, h_{i,t} + s_{i} \, \eta_{i, t+1},
    \qquad
    i \in \{z, c, \lambda\}.
\end{equation*}

The consumption and preference shock growth rates are
%
\begin{equation}\label{eq:ssygc}
    g_{c, t+1}
    = \mu_c + z_t + \sigma_{c, t} \, \xi_{c, t+1} 
    \quad \text{and} \quad
    g_{\lambda, t+1} = h_{\lambda, t+1}.
\end{equation}
%
The innovations are all independent and standard normal.  

In this setting, the operator $\HH$ takes the form
%
\begin{equation*}
    (\HH f)(x)
    = \EE_x
        \, f(X_{t+1}) 
        \exp \left\{
            \theta h_{\lambda, t+1}
            + (1-\gamma)(\mu_c + z_t + \sigma_{c, t} \xi_{c, t+1})
        \right\}.
\end{equation*}
%
We can reduce dimensionality in the conditional expectation by integrating out
the independent innovation $\xi_{c, t+1}$, which leads to 
%
\begin{equation}
    (\HH f)(x)
    = \exp \left\{
            (1-\gamma)(\mu_c + z)
            + \frac{1}{2} (1-\gamma)^2 \sigma_c^2 
        \right\}
        \EE_x \, f(X_{t+1}) 
        \exp \left\{
            \theta h_{\lambda, t+1}
        \right\},
\end{equation}
%
where the conditioning is on $X_t = x$ as given in \eqref{eq:state}.
We can write the operator more explicitly as
%
\begin{equation}
    (\HH f)(x)
    = \kappa(z, h_c)
        \EE_x \, f(X_{t+1}) 
        \exp \left\{
            \theta h_{\lambda, t+1}
        \right\},
\end{equation}
%
where
%
\begin{equation}\label{eq:kap}
    \kappa(z, h_c) 
     := \exp \left\{
            (1-\gamma)(\mu_c + z)
            + \frac{1}{2} (1-\gamma)^2 
            [\phi_c \, \bar{\sigma} \, \exp(h_c)]^2 
        \right\}.
\end{equation}
%
The baseline parameter values can be found in the code repository.



\subsection{Example: The GCY Case}

In this section, we analyze the stability properties of the model of
\cite{GomezYaron2020}. The authors add inflation dynamics to a long-run risk
model similar to the one of \cite{schorfheide2018identifying}. In particular,
they assume that the expected inflation rate $z_{\pi,t}$ affects the mean growth
rate of consumption:
%
\begin{equation}\label{eq:gcycg}
	g_{c, t+1} 
	= \ln \left( \frac{C_{t+1}}{C_t} \right)    
	= \mu_c + z_t + \sigma_{c, t} \, \xi_{c, t+1},
\end{equation}
%
where
%
\begin{align*}
	z_{t+1} 
	&= \rho \, z_t + \rho_{\pi} \, z_{\pi,t}+ 
	\sigma_{z, t} \, \eta_{t+1}\\
	z_{\pi,t+1} 
	&= \rho_{\pi \pi} \, z_{\pi,t} + \sigma_{z \pi, t} \, \eta_{\pi,t+1}
\end{align*}
%
and
%
\begin{equation}
	\sigma_{i,t} 
	= \phi_i \, \bar{\sigma} \, \exp(h_{i, t})
\end{equation}
%
with
%
\begin{equation*}
	h_{i, t+1}
	= \rho_i \, h_{i,t} + s_i \, \eta_{i, t+1}
	\quad \text{for } i \in \{z, c, z\pi \}.
\end{equation*}
%
Note that also the expected inflation rate $z_{\pi,t}$ has stochastic volatility
$\sigma_{z\pi,t}$. As in the model of \cite{schorfheide2018identifying}, the
process $\{\lambda_t\}$ follows \eqref{eq:ssygc} and all shocks are
\textsc{iid} and standard normal. Hence, the state vector $x$ contains 6 states
and is given by
%
\begin{equation*}
    x = (z, z_\pi, h_z, h_c, h_{z\pi}, h_\lambda) \in \XX := \RR^6. 
\end{equation*}
%
We can apply the same conditioning as in the model of
\cite{schorfheide2018identifying} and we need to compute $\sS_c$ numerically
again. The baseline parameter values can be found in the code repository.



\section{Solution Method}

Background on Newton's algorithm in $\RR^N$.  Let $f$ be a smooth map from
$\RR^N$ to itself.  We want to find the $x \in \RR^N$ that solves $f(x)=x$.
Ordinary successive approximation uses
%
\begin{equation}
    x_{k+1} = f(x_k)
\end{equation}
%
Newton's method first sets $g(x) = f(x) -x$, so that we are seeking a root $x$
satisfying $g(x)=0$, and then iterates on
%
\begin{equation}
    x_{k+1} = g(x_k) + J(x_k)^{-1} g(x_k)
\end{equation}
%
where $J(x)$ is the Jacobian of $g$ at $x$.


\subsection{Discretization}

For computation, the state process is distcretized onto a state space $S =
\{x_1, \ldots, x_N\}$ of size $N \in \NN$ and the operator $\HH$ is
represented by a matrix $\bH$, with 
%
\begin{equation}\label{eq:defkb}
    \bH(n, n') = \sum_{n'=1}^N 
        \,
        \exp
        \left\{ 
            \theta g_\lambda(x_n, x_{n'}, \xi) 
            + (1-\gamma) g_c(x_n, x_{n'}, \xi)
        \right\}
    \bP(n, n'),
\end{equation}
%
where $\bP$ is an $N \times N$ matrix with  $\bP(n, n')$ representing the
probability that the discretized state process transitions from state $x_n$ to
state $x_{n'}$ in one unit of time.

The discretization of $\TT$ is written as $\bT$ and the problem is to find the
fixed point of 
%
\begin{equation}\label{eq:wcopd}
    (\bT w) = 1 + \beta \,  (\bH w^\theta)^{1/\theta}
\end{equation}
%
in the set of strictly positive vectors in $\RR^N$.





\bibliographystyle{ecta}

\bibliography{localbib}


\end{document}
