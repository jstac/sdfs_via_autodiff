\documentclass[12pt, reqno]{amsart}

%%%%%%%%%%%%%%%%%%%%%  MY STUFF %%%%%%%%%%%%%%%%%%%%%%%%%%%%%%%%%%
%\makeatletter
%\g@addto@macro{\endabstract}{\@setabstract}
%\makeatother


%\usepackage{epsfig}
\usepackage{graphics, stackrel}
\usepackage{amsmath, amssymb, amsthm}
\usepackage{graphicx}
\usepackage{verbatim}
\usepackage{amsfonts}
\usepackage{natbib}
\usepackage{enumitem}
%font
%\usepackage{lmodern}
\usepackage[T1]{fontenc}
%\usepackage{mathpazo}
%\usepackage{tgpagella}

%subfloats / figures
\usepackage{caption}
\usepackage{subcaption}

% For pandas latex tables
\usepackage{booktabs}


\usepackage{fancyvrb}
\usepackage[usenames,dvipsnames,svgnames,table]{xcolor}
\usepackage{mdwlist}

\usepackage[citecolor=blue, colorlinks=true, linkcolor=blue]{hyperref}


% lists
\usepackage{enumitem}
\setlist[enumerate]{itemsep=2pt,topsep=3pt}
\setlist[itemize]{itemsep=2pt,topsep=3pt}
\setlist[enumerate,1]{label=(\alph*)}

\usepackage{mathrsfs}  % caligraphic
%\usepackage{stix}  % caligraphic
\usepackage{bbm}
\usepackage{bm}        % bold symbols


%% page layout
\usepackage[left=1.25in, right=1.25in, top=1.25in, bottom=1.25in, includehead, includefoot]{geometry}

% nice inequalities
\renewcommand{\leq}{\leqslant}
\renewcommand{\geq}{\geqslant}

% inner product
\providecommand{\inner}[1]{\left\langle{#1}\right\rangle}
\providecommand{\innerp}[1]{\left\langle{#1}\right\rangle_\pi}


\usepackage[ruled, linesnumbered]{algorithm2e}

%extra spacing
\renewcommand{\baselinestretch}{1.25}

%horizonal line
\newcommand{\HRule}{\rule{\linewidth}{0.3mm}}

% skip a line between paragraphs, no indentation
\setlength{\parskip}{1.5ex plus0.5ex minus0.5ex}
\setlength{\parindent}{0pt}

% footnote without a maker (blfootnote)
\newcommand\blfootnote[1]{%
  \begingroup
  \renewcommand\thefootnote{}\footnote{#1}%
  \addtocounter{footnote}{-1}%
  \endgroup
}

\DeclareMathOperator{\Span}{span}
\DeclareMathOperator{\diag}{diag}
\DeclareMathOperator*{\argmin}{arg\,min}
\DeclareMathOperator*{\argmax}{arg\,max}
\DeclareMathOperator{\sign}{sign}
\DeclareMathOperator{\cl}{cl}
%\DeclareMathOperator{\overset{\circ}}{int}
\DeclareMathOperator{\Prob}{Prob}
\DeclareMathOperator{\determinant}{det}
\DeclareMathOperator{\Var}{Var}
\DeclareMathOperator{\Cov}{Cov}
\DeclareMathOperator{\graph}{graph}

% mics short cuts and symbols
\newcommand{\st}{\ensuremath{\ \mathrm{s.t.}\ }}
\newcommand{\setntn}[2]{ \{ #1 : #2 \} }
\newcommand{\fore}{\therefore \quad}
\newcommand{\preqsd}{\preceq_{sd} }
\newcommand{\toas}{\stackrel {\textrm{ \scriptsize{a.s.} }} {\to} }
\newcommand{\tod}{\stackrel { d } {\to} }
\newcommand{\tou}{\stackrel { u } {\to} }
\newcommand{\toweak}{\stackrel { w } {\to} }
\newcommand{\topr}{\stackrel { p } {\to} }
\newcommand{\disteq}{\stackrel { \mathscr D } {=} }
\newcommand{\eqdist}{\stackrel {\textrm{ \scriptsize{d} }} {=} }
\newcommand{\iidsim}{\stackrel {\textrm{ {\sc iid }}} {\sim} }
\newcommand{\1}{\mathbbm 1}
\newcommand{\la}{\langle}
\newcommand{\ra}{\rangle}
\newcommand{\dee}{\,{\rm d}}
\newcommand{\og}{{\mathbbm G}}
\newcommand{\ctimes}{\! \times \!}
\newcommand{\sint}{{\textstyle\int}}

\newcommand{\given}{\, | \,}
\newcommand{\A}{\forall}

% d for integrals
\newcommand*\diff{\mathop{}\!\mathrm{d}}
\newcommand*\e{\mathrm{e}}


% Special symbols and shortcuts
\newcommand{\bmeta}{\bm{\eta}}
\newcommand{\bmxi}{\bm{\xi}}

\newcommand{\infot}{\fF_t}

\newcommand{\pspace}{\mathscr{P}(\mathsf{X})}
\newcommand{\cspace}{\mathscr{C}(\mathsf{X})}

%\renewcommand{\times}{\! \times \!}

\newcommand{\aA}{\mathscr A}
\newcommand{\cC}{\mathscr C}
\newcommand{\dD}{\mathscr D}
\newcommand{\bB}{\mathscr B}
\newcommand{\oO}{\mathcal O}
\newcommand{\gG}{\mathscr G}
\newcommand{\hH}{\mathcal H}
\newcommand{\kK}{\mathcal K}
\newcommand{\iI}{\mathcal I}
\newcommand{\mM}{\mathcal M}
\newcommand{\eE}{\mathcal E}
\newcommand{\fF}{\mathscr F}
\newcommand{\qQ}{\mathcal Q}
\newcommand{\tT}{\mathcal T}
\newcommand{\xX}{\mathcal X}
\newcommand{\yY}{\mathcal Y}
\newcommand{\wW}{\mathcal W}
\newcommand{\uU}{\mathcal U}
\newcommand{\lL}{\mathcal L}
\newcommand{\rR}{\mathcal R}
\newcommand{\zZ}{\mathcal Z}

\newcommand{\sS}{\mathscr S}

\newcommand{\pP}{\mathcal P}

\newcommand{\vV}{\mathcal V}

\newcommand{\BB}{\mathbbm B}
\newcommand{\RR}{\mathbbm R}
\newcommand{\CC}{\mathbbm C}
\newcommand{\QQ}{\mathbbm Q}
\newcommand{\NN}{\mathbbm N}
\newcommand{\GG}{\mathbbm G}
\newcommand{\UU}{\mathbbm U}
\newcommand{\YY}{\mathbbm Y}
\newcommand{\ZZ}{\mathbbm Z}
\newcommand{\KK}{\mathbbm K}
\newcommand{\PP}{\mathbbm P}
\newcommand{\EE}{\mathbbm E}
\newcommand{\TT}{\mathbbm T}

\newcommand{\var}{\mathbbm V}

\newcommand{\II}{\mathbb I}
\newcommand{\WW}{\mathbb W}

\renewcommand{\phi}{\varphi}

\newcommand{\XX}{\mathsf X}

\newcommand{\bP}{\mathbf P}
\newcommand{\bQ}{\mathbf Q}
\newcommand{\bE}{\mathbf E}
\newcommand{\bM}{\mathbf M}
\newcommand{\bX}{\mathbf X}
\newcommand{\bY}{\mathbf Y}

\theoremstyle{plain}
\newtheorem{theorem}{Theorem}[section]
\newtheorem{corollary}[theorem]{Corollary}
\newtheorem{lemma}[theorem]{Lemma}
\newtheorem{proposition}[theorem]{Proposition}


\theoremstyle{definition}
\newtheorem{definition}{Definition}[section]
\newtheorem{axiom}{Axiom}[section]
\newtheorem{example}{Example}[section]
\newtheorem{remark}{Remark}[section]
\newtheorem{notation}{Notation}[section]
\newtheorem{assumption}{Assumption}[section]
\newtheorem{condition}{Condition}[section]


%\DeclareTextFontCommand{\emph}{\bfseries}

%%%%%%%%%%%%%%%%%% end my preamble %%%%%%%%%%%%%%%%%%%%%%%%%%%%%%%%%%%






\begin{document}


\title{}




\begin{center}
    \LARGE 
    Recursive Utility with Preference Shocks: \\
    Existence and Uniqueness\blfootnote{The authors are grateful for research
    assistance from Shu Hu, financial support from ARC grant FT160100423, and
valuable feedback from seminar participants at Xiamen and John Hopkins
Universities.  The usual disclaimer applies.}
 
    \vspace{1em}

    \large
    John Stachurski,\textsuperscript{a}
    Ole Wilms\textsuperscript{b} 
    and
    Junnan Zhang\textsuperscript{c} 
    \par \bigskip

    \small
    \textsuperscript{a} Research School of Economics, Australian National University \\
    \textsuperscript{b} Universit\"at Hamburg and Tilburg University  \\
    \textsuperscript{c} Center for Macroeconomic Research at School of
    Economics, and Wang Yanan Institute for Studies in Economics, Xiamen
    University \\ \bigskip

    \normalsize
    \today
\end{center}


\begin{abstract}
    This paper studies existence and uniqueness
    of recursively defined utility in asset pricing models with preference shocks. We
    provide conditions that clarify existence and uniqueness for a wide range of models,
    including exact necessary and sufficient conditions
    for the most standard formulations.  The conditions isolate the roles of preference
    parameters, as well as the different risks that drive the
    consumption and preference shock processes. One notable finding is that,
    for agents who prefer early resolution of risk, introducing preference shocks
    always relaxes the existence condition. This effect is 
    large in common calibrations, so the presence of preference shocks makes
    the existence of a solution significantly more likely. \vspace{1em}

    \noindent
    \textit{JEL Classifications:} D81, G11 \\
    \textit{Keywords:} Asset pricing, recursive preferences, preference shocks, Epstein--Zin preferences, long-run risk
\end{abstract}



\maketitle

\pagebreak
%section
\section{Introduction}

Over the past decade, models that combine recursive preferences with
preference shocks---often interpreted as demand shocks---have found a central
place in the debate on how to reconcile observed asset price dynamics with
time paths for dividends and other cash flows.  For many quantitative models,
both recursive preferences and preference shocks are crucial to the generation of
accurate time paths for prices and returns.  A sample of the literature can be
found in \cite{Albuquerque2016}, \cite{basu2017uncertainty},
\cite{schorfheide2018identifying}, \cite{chen2019search},
\cite{GomezYaron2020}, \cite{deGroot2018}, \cite{creal2020bond}, and
\cite{degroot2021valuation}.  In all of this work, recursive preferences are
formulated via the Epstein--Zin specification.

The success of this framework comes at the cost of substantial nonlinearities
in expressions for lifetime utility, as well as for the equilibrium
wealth-consumption ratio (which is central to the analysis of prices and
returns, since it controls the stochastic discount factor that maps cash flows
to prices).  As a result of these nonlinearities, it has hitherto been an open
question as to whether or not the models described above have well-defined
solutions at the stated parameterizations.

%Existence and uniqueness in models with recursive preferences is a widely
%discussed topic, see for example \cite{epstein1989risk},
%\cite{marinacci2010unique}, \cite{hansen2012recursive},
%\cite{marinacci2019unique}, \cite{Pohl2019}, and
%\cite{borovicka2020necessary}. However, none of these papers allows for
%preference shocks in the utility specification. \textcolor{red}{The previous
  %two sentences are basically the same as ``Our paper builds on...'' on page
  %4. Maybe this paragraph should just starts with how preference shocks are
  %important?} As such shocks are crucial to explain asset price dynamics, see
%\cite{Campbell1986}, \cite{Albuquerque2016},
%\cite{schorfheide2018identifying}, \cite{chen2019search} and
%\cite{GomezYaron2020}, settling the question of existence and uniqueness is
%of particular importance. 


This paper provides sufficient conditions for existence and uniqueness that
are general enough to cover a wide range of models, including all of those
listed above.  Moreover, for standard formulations of the preference shock
that are commonly used in asset pricing models, we provide exact necessary and
sufficient conditions. We also show that, when the conditions are violated,
the models make no usable predictions (i.e., existence fails, rather than
uniqueness).

We find that the delineation between existence and nonexistence hinges on a
stability coefficient that depends on the spectral radius of a valuation operator
relative to three parameters appearing in Epstein--Zin preferences.  To
facilitate calculation and interpretation of the stability coefficient, we use
a local spectral radius theorem to show that, under a weak independence
assumption which holds in many applications, the stability coefficient can be
decomposed into three terms. These three terms depend on the rate of time
preference, the time path for preference shocks and the dynamics of
consumption growth respectively.  In this manner, we are able to isolate and
separately analyze the roles of time preference, preference shocks and
consumption dynamics.

Through this decomposition, we show that, when intertemporal
elasticity of  substitution (IES) is greater than one---as is commonly assumed in
the asset pricing literature---risk aversion with respect to atemporal gambles
implies that nontrivial preference shocks always loosen the existence
condition. Intuitively, this is because, after adjusting for intertemporal
elasticity of substitution, preference shocks can be seen as adding volatility
to consumption.  Since risk averse agents dislike volatile consumption
flows, nontrivial preference shocks suppress the wealth-consumption ratio,
which acts against divergence. For risk averse agents with an IES smaller than
one, the logic works in reverse. In this case, the income effect dominates the
substitution effect, so agents choose a higher wealth-consumption ratio when
preference shocks are added to the model. Hence preference shocks
increase the likelihood that no stable solution exists.\footnote{To be more
    specific, we show that preference shocks loosen the existence condition if
    and only if the IES and the risk aversion coefficient are both greater
    than one or both less than one. Incidentally, these are the two cases in
which preference shocks can help explain the equity premium puzzle as shown by
\cite{Albuquerque2016}.}


We consider four applications to showcase our approach: the benchmark model of
\cite{Albuquerque2016} with {\sc iid} normal consumption growth, a standard
long-run risk model as in \cite{bansal2004risks} with preference shocks, the long-run risk model with
multiple log-normal volatility processes of \cite{schorfheide2018identifying}
and the long-run risk model with inflation expectations and multiple
log-normal volatility processes of \cite{GomezYaron2020}. In all cases,
preference shocks are independent of consumption growth and follow
an AR(1) process. This allows us to isolate the effect of preference shocks
on existence and uniqueness of recursive utility. For
commonly used parameterizations of the models, the influence of preference
shocks on existence is large and, in particular, it significantly dominates
the effect of consumption risks. This implies that, for any subjective discount
factor smaller than unity, a solution does exist. Furthermore, highly persistent and
volatile preference shocks significantly relax the existence condition, so
that a solution exists even for rapidly growing economies.

We also show that if the IES approaches 1 from above, the stability
coefficient diverges to minus infinity. This implies that for an IES
sufficiently close but larger than 1, a solution always exists. In contrast, if
the IES approaches 1 from below, the stability coefficient diverges to
infinity and we have non-existence. These results are in line with
\cite{deGroot2018}, who show that the influence of preference shocks on model
outcomes can become arbitrarily large for an IES close to one. In response,
\cite{degroot2021valuation} use an alternative utility specification to
include preference shocks into an asset pricing model with Epstein-Zin
preferences. We provide sufficient conditions for existence and uniqueness for
this specification as well. 

Our paper builds on a large literature that deals with the problem
of existence and uniqueness of recursive utility, with prominent examples
including \cite{epstein1989risk}, \cite{boyd1990recursive},
\cite{marinacci2010unique}, \cite{hansen2012recursive},
\cite{bauerle2018stochastic}, \cite{becker2018recursivei},
\cite{becker2018recursiveii}, \cite{marinacci2019unique}, \cite{Pohl2019},
\cite{borovicka2020necessary}, \cite{balbus2020recursive},
\cite{bloise2021not} and \cite{christensen2022existence}. However, none of
these papers considers preference shocks. This paper explicitly includes
preference shocks in the utility specification and focuses directly on the
connection between preference shocks and existence and uniqueness of
recursive utility. As our approach does not require strong assumptions on the
underlying consumption process, it can also handle cases in which preference
shocks are added to models that rely on other risk factors such as models
with consumption disasters as in \cite{Barro2009} and \cite{Wachter2013},
models with volatility of volatility as in \cite{Bollerslev2009} and
\cite{Bollerslevetal2015} and models with jumps in volatility and growth
rates as in \cite{DrechslerYaron11}.


%There are many papers dealing with the problem
%of existence and uniqueness of recursive utility, with prominent examples
%including \cite{epstein1989risk}, \cite{boyd1990recursive},
%\cite{marinacci2010unique}, \cite{hansen2012recursive},
%\cite{bauerle2018stochastic}, \cite{becker2018recursivei},
%\cite{becker2018recursiveii}, \cite{marinacci2019unique}, \cite{Pohl2019},
%\cite{borovicka2020necessary}, \cite{balbus2020recursive},
%\cite{bloise2021not} and \cite{christensen2022existence}. However, none of
%these papers considers preference shocks. In contrast, this paper explicitly
%includes preference shocks in the utility specification and focuses directly
%on the connection between preference shocks and existence and uniqueness of
%recursive utility.
%
%In this paper we build on earlier work by \cite{borovicka2020necessary} to
%provide tight conditions for existence and uniqueness of recursive utilities,
%as well as for the wealth-consumption ratio.  While
%\cite{borovicka2020necessary} failed to encompass preference shocks, we
%include them directly, under alternative specifications.  In all cases, we
%provide sufficient conditions for existence and uniqueness.  In the most
%standard formulation, we provide exact necessary and sufficient conditions.
%We also show that, when the conditions are violated, it is existence that
%fails, rather than uniqueness.

%The references \cite{deGroot2018} and \cite{degroot2021valuation} mentioned
%above are closely related to our research.  The focus of these papers is on
%stability and interpretation of Epstein--Zin recursive utility in an asset
%pricing setting under different specifications of preference shocks.
%However, they do not provide explicit results on existence and uniqueness of
%recursive utility, which is the primary focus of this paper.

Our work is also related to \cite{stachurski2021dynamic}, who study dynamic
programming with state-dependent discounting. Although preference shocks can
be treated as state-dependent discounting, their focus is on the dynamic
programming problem in which consumption of the agent is endogenous. We also
develop new analytical tools in this paper that allow for exact necessary and
sufficient conditions under a full set of parameters, while
\cite{stachurski2021dynamic} give only sufficient conditions under a
restricted set of parameters. As a result, we are able to settle the issue of
existence and uniqueness for all standard-form asset pricing models with preference shocks
considered in the applications.

The rest of the paper is structured as follows. Section~\ref{s:env} sets out
the model. Section~\ref{s:results} provides our main results.
Section~\ref{stab_coef} decomposes the stability coefficient.
Section~\ref{s:app} discusses applications. Section~\ref{s:alt_spec} studies
the alternative specification proposed in \cite{deGroot2018}.
Section~\ref{s:conclusion} concludes. The proofs are in the appendix.

\section{Environment}\label{s:env}

We consider an endowment economy containing a representative agent with
Epstein-Zin preferences (see~\cite{epstein1989risk}
and~\cite{weil1990nonexpected}). We begin by introducing preference shocks
using the formulation of \cite{Albuquerque2016} and
\cite{schorfheide2018identifying}, where preferences are defined recursively
by
%
\begin{equation}
    \label{eq:agg}
    V_t = \left[
            (1 - \beta) \lambda_t C_t^{1-1/\psi}
            + \beta \left\{ \rR_{t, 1-\gamma} \left(V_{t+1}
            \right) \right\}^{1-1/\psi}
          \right]^{1/(1-1/\psi)}.
\end{equation}
%
Here 
%
\begin{itemize}
    \item $\{ C_t \}_{t \geq 0}$ is a consumption path,
    \item $\{ \lambda_t \}_{t \geq 0}$ is a sequence of preference shocks,
    \item $V_t$ is the utility value of the path extending on from time $t$ and 
    \item $\rR_{t, 1-\gamma}$ is the Kreps--Porteus certainty equivalent operator
        conditional on time $t$ information, defined by
    %
    \begin{equation}
        \label{eq:ce}
        \rR_{t, 1-\gamma}(V_{t+1})
        = ( \EE_t  V^{1-\gamma}_{t+1} )^{1/(1-\gamma)}.
    \end{equation}
    %
\end{itemize}

The parameter $\beta \in (0, 1)$ is a time discount factor, while $\gamma
\not=1$ governs risk aversion and $\psi \not=1$ is the elasticity of
intertemporal substitution.  We take the consumption stream and sequence of
preference shocks as given and seek a solution for utility $V_t$. Consumption
growth and the growth rate of the preference shock are given by the generic
formulas
%
\begin{equation}
    \label{eq:kappa}
    \ln \left( \frac{C_{t+1}}{C_t} \right)
    = g_c(X_t, X_{t+1}, \xi_{t+1})
    \quad \text{and} \quad
    \ln \left( \frac{\lambda_{t+1}}{\lambda_t} \right) 
    = g_\lambda(X_t, X_{t+1}, \xi_{t+1}),
\end{equation}
%
where 
%
\begin{itemize}
    \item $\{ X_t \}_{t \geq 0}$ is a discrete time Markov process on a compact
        metric space $\XX$,
    \item $\{ \xi_t \}_{t \geq 1}$ is an {\sc iid} process supported on $\YY \subset \RR^k$, and
    \item $g_i \colon \XX \times \XX \times \YY \to \RR$ is continuous for
        each $i \in \{c, \lambda\}$.
\end{itemize}
%
The processes $\{X_t\}$  and $\{\xi_t\}$ are assumed to be independent.

\begin{remark}
    The assumption that typically binds in the list given above is compactness of
    the state space $\XX$.  However, for stationary models, this assumption holds if we truncate
    innovations to lie within any given interval. Moreover, all
    applications that we consider here use standard normal innovations (although the
    resulting state process can, of course, be non-Gaussian).  If all standard
    normal innovations are truncated to lie in the interval $[-k, k]$ then,
    for any form of quantitative analysis, the exogenous state process
    becomes indistinguishable from the original whenever $k \geq 10$,
    say.\footnote{There is no need for---and we do not use---explicit
        truncation in computational analysis, since every numerical
        implementation inherently truncates to the range of 64 bit
        floats.}
\end{remark}

The state process $\{ X_t\}$ updates according to transition density $q$, in
the sense that
%
\begin{equation}\label{eq:defq}
    \PP\{X_{t+1} \in B \given X_t = x\}
    = \int_B q(x, y) \, \diff y
\end{equation}
%
for every $x \in \XX$ and Borel set $B \subset \XX$.
In applications, either $\XX$ is a subset of $\RR^j$, and integration is with
respect to Lebesgue measure, or $\XX$ is discrete. In the second case, we
understand the integral in \eqref{eq:defq} as a sum, and $q$ is a transition
matrix. Let $q^n$ denote the $n$-step transition density, which gives transition
probabilities over $n$ units of time.
%
\begin{assumption}\label{a:q}
    The process $\{ X_t \}_{t \geq 0}$ is stationary, with $X_t \eqdist \pi$ for
    all $t \geq 0$.  The transition density $q$ is continuous on $\XX \times
    \XX$ and there exists an $\ell \in \NN$ such that $q^\ell$ is everywhere
    positive.
\end{assumption}
%
In computer implementations, where $\XX$ is necessarily finite,
Assumption~\ref{a:q} is equivalent to the statement that $q$ is aperiodic and
irreducible.\footnote{In the discrete setting, we adopt the discrete topology,
so $q$ and all other functions on the state are automatically continuous.}


\section{Geometric Stability}\label{s:results}

In this section we present our main results on the solution of \eqref{eq:agg}.

\subsection{Set Up}

In order to obtain a stationary solution for recursive utility, our first step
is to normalize $\{V_t\}$ by dividing out the growing components $\{C_t\}$ and
$\{\lambda_t\}$.  This leads us to introduce the transformed variable
%
\begin{equation}\label{eq:gdef}
    G_t := \frac{1}{\lambda_t^\theta} \left( \frac{V_t}{C_t} \right)^{1-\gamma}
    \quad \text{where} \quad
    \theta := \frac{1-\gamma}{1-1/\psi} .
\end{equation}
%
Although this particular normalization is not a standard choice, it
possesses one major advantage: when the recursion for utility is expressed in
terms of $\{G_t\}$ instead of $\{V_t\}$, the 
nonlinear components of the evaluation and linear components naturally
separate.  In our main results,  we exploit this
separation to obtain sharp conditions for existence and uniqueness of
solutions.

To see how this works, we 
divide \eqref{eq:agg} by $\lambda_t C_t^{1-1/\psi}$
and use the definition of $G_t$ to rewrite 
\eqref{eq:agg} as
%
\begin{equation}\label{eq:agg2}
    G_t = 
    F 
    \left[ 
        \,
        \EE_t \, G_{t+1}  \,
        \exp
        \left\{ 
            \theta g_\lambda(X_t, X_{t+1}, \xi_{t+1})
            + (1-\gamma) g_c(X_t, X_{t+1}, \xi_{t+1})
        \right\}
        \,
    \right],
\end{equation}
%
where
%
\begin{equation*}
    F(t) :=
        \left(
            1-\beta + \beta \, t^{1/\theta} 
        \right)^\theta 
        \qquad (t \geq 0).
\end{equation*}
%

If $\theta < 0$, we set $F(0)=0$, which makes $F$ continuous.
Figure~\ref{f:w} shows that $F$ is either concave increasing or convex
increasing, depending on the value of $\theta$.\footnote{In the figure we set
$\beta = 0.5$.  Any $\beta \in (0,1)$ produces the same basic shape properties.}
%
\begin{figure}
    \centering
    \includegraphics[width=0.7\textwidth]{figs/w.pdf}
    \caption{\label{f:w} Shape properties of $F$.}
\end{figure}
%
Evidently, any solution $\{ G_t \}$ to \eqref{eq:agg2} yields a solution $\{
 V_t \}$  to the original utility problem \eqref{eq:agg}, which can be
 obtained by unwinding the transformation in \eqref{eq:gdef}. Note that by
 assuming $\psi \neq 1$, we rule out  cases where $\theta \to \pm \infty$, as
 $\{ V_t \}$ is not defined for $\psi \neq 1$ (see \cite{deGroot2018}). 

\subsection{Existence and Uniqueness}

To study solutions to \eqref{eq:agg2}, we take $\cC$ to be the set of continuous
everywhere positive functions on $\XX$.  In what follows, a \emph{stationary
Markov solution} to \eqref{eq:agg2} is a $g \in \cC$ such that the stochastic
process $\{G_t\} := \{g(X_t)\}$ satisfies \eqref{eq:agg2} with probability one
for all $t \geq 0$.  For any operator $\TT \colon \cC \to \cC$, we call $\TT$
\emph{globally geometrically stable} on $\cC$ if $\TT$ has a unique fixed point $g^*
\in \cC$ and, for all $g \in \cC$, there exists an $a <
1$ and $N < \infty$ such that
%
\begin{equation}
    \sup_{x \in \XX} | (\TT^n g)(x) - g^*(x) |
    \leq a^n N
    \;
    \text{ for all } n \in \NN.
\end{equation}
%

Let $\KK$ be the linear operator defined by
%
\begin{equation}\label{eq:defk}
    (\KK g)(x) = \EE_x \left[
        \, g(X_{t+1})  \, \Gamma(X_t, X_{t+1}, \xi_{t+1})
    \right]
    \qquad (x \in \XX),
\end{equation}
%
where $\EE_x$ conditions on $X_t = x$ and we use the notation
%
\begin{equation}\label{eq:sdef}
    \Gamma(x, y, \xi) 
    :=  \exp
    \left\{ 
        \theta g_\lambda(x, y, \xi) + (1-\gamma) g_c(x, y, \xi)
    \right\}.
\end{equation}
%
The operator $\KK$ applies a form of linear discounting to rewards one period
in the future. In order for $\KK$ to be well defined, we assume that $\EE_\xi
\Gamma(x, y, \xi)$ is finite for all $x, y \in \XX$, where $\EE_\xi$ denotes
the expectation with respect to $\xi$.\footnote{In other words, the
  discounting is expected to be finite, which is a rather weak assumption and
  is satisfied in all the applications we consider.}

Let $\TT
\colon \cC \to \cC$ be the (modified) \emph{Koopman operator} defined by $\TT =
F \circ \KK$.  More explicitly,
%
\begin{equation}\label{eq:deft}
    (\TT g)(x) = F [ (\KK g)(x)]
    \qquad (g \in \cC, \; x \in \XX).
\end{equation}
%
By construction, $g \in \cC$ is a stationary Markov solution to
\eqref{eq:agg2} if and only if $g$ is a fixed point of $\TT$.
As we show in the appendix, $\TT$ is monotone increasing and either
convex or concave, depending on the value of $\theta$.  As a result, we can
analyze its fixed points and stability properties using the theory of monotone
concave operators and monotone convex operators.\footnote{For an overview of
    the literature, see \cite{zhang2013}, or the earlier work by
    \cite{krasnoselskii1964}.  The specific fixed point result that we use
    below is a modified version of a theorem due to \cite{du1990fixed}.}

The basic intuition can already be seen in Figure~\ref{f:w}, since $F$ has
these same properties (i.e., monotonicity and convexity or concavity).  The
map $F$ has a unique fixed point in the interior of its domain whenever $F(t)
> t$ for small positive $t$ and $F(t) < t$ for all sufficiently large $t$.  It
is also clear from the figures that, when these boundary conditions are
satisfied,  $F^n(t)$ converges to the unique fixed point as $n \to \infty$ for
any $t > 0$.

While this intuition speaks only to $F$, the operator $\KK$ is linear and
monotone increasing, so the composition $\TT = F \circ \KK$ inherits the
monotonicity and convexity (or concavity) properties of $F$.  As a result,
existence of a unique stationary Markov solution depends on an analogous set
of boundary conditions for $\TT$, where small functions are mapped strictly up
and large functions are mapped strictly down.  Whether or not these boundary
conditions hold depends both on the parameters in $F$ and the properties of
the operator $\KK$.  

Regarding the properties of $\KK$ it turns out that, to analyze these boundary
conditions, it is enough to study the spectral radius of $\KK$, which we denote
by $r(\KK)$.\footnote{The appendix clarifies the definition of the spectral
radius used here and below.} In stating our main theorem, we let
%
\begin{equation}\label{eq:blam}
    \sS := \ln \beta + \frac{\ln r(\KK)}{\theta}.
\end{equation}
%

The content of the next theorem is that the boundary conditions for $\TT$
discussed above hold if and only if $\sS < 0$.

\begin{theorem}\label{t:bk}
    If Assumption~\ref{a:q} holds, then the following statements are equivalent:
    %
    \begin{enumerate}
        \item[{\rm (a)}] $\sS < 0$.
        \item[{\rm (b)}] $\TT$ is globally geometrically stable on $\cC$.  
    \end{enumerate}
    %
    Moreover, if $\sS \geq 0$, then no stationary Markov solution exists in
    $\cC$.
\end{theorem}

Theorem~\ref{t:bk} provides an exact dichotomy. If $\sS < 0$,  then a unique
and globally attracting stationary Markov solution $g^*$ exists in $\cC$.  On
the other hand, if $\sS < 0$ fails, then not only does global geometric
stability fail, but existence fails, specifically.  


\subsection{The Wealth-Consumption Ratio}

Next we study the wealth-consumption ratio, the significance of which was
discussed in the introduction.
In this model, the equilibrium wealth-consumption ratio $w(X_t) = W_t / C_t$
obeys
%
\begin{equation*}
    \beta^{\theta}
    \EE_t
    \left[
    \left( \frac{\lambda_{t+1}}{\lambda_t} \right)^\theta
        \left( \frac{C_{t+1}}{C_t} \right)^{1-\gamma}
        \left( \frac{w(X_{t+1})}{w(X_t)-1} \right)^\theta
    \right] = 1
\end{equation*}
%
(see, e.g.,  \cite{schorfheide2018identifying}).  

Rearranging the previous expression gives
%
\begin{equation*}
    (w(X_t)-1)^\theta
    = \beta^{\theta}
    \EE_t
    \left[
    \left( \frac{\lambda_{t+1}}{\lambda_t} \right)^\theta
        \left( \frac{C_{t+1}}{C_t} \right)^{1-\gamma}
        w(X_{t+1})^\theta
    \right].
\end{equation*}
%
Conditioning on $X_t = x$ and writing pointwise on $\XX$ yields $w = 1 +
\beta \left( \KK w^\theta \right)^{1/\theta}$.  A function $w \in \cC$ solves
this equation if and only if $w$ is a fixed point of the operator $\UU \colon
\cC \to \cC$ defined by 
%
\begin{equation}\label{eq:wcop}
    (\UU w) = 1 + \beta \,  (\KK w^\theta)^{1/\theta}.
\end{equation}
%

By proving a topological conjugacy relationship between $\UU$ and $\TT$, we
establish the following result.

\begin{proposition}\label{p:wcr}
    $\UU$ has the same stability properties as $\TT$.  In particular, when
    Assumption~\ref{a:q} holds, $\UU$ is globally geometrically stable on
    $\cC$  if and only if $\sS < 0$.
\end{proposition}

% In practice, in the applications below, we iterate with $\UU$ rather than
% $\TT$, since our main interest is in computing the wealth-consumption
% ratio. 

Inspecting the relationship between the wealth-consumption ratio and
recursive utility also sheds light on the existence issue discussed in the
previous section. For this model, the wealth-consumption ratio satisfies
%
\begin{equation}\label{eq:wc}
    w(X_t) = \frac{W_t}{C_t} 
    % = \frac{1}{1-\beta} 
    % \frac{1}{\lambda_t} \left(\frac{V_t}{C_t}\right)^{1-\frac{1}{\psi}} 
    = \frac{1}{1-\beta} g(X_t)^{1/\theta},
\end{equation}
%
or equivalently, $g(X_t) = (1-\beta)^\theta w(X_t)^\theta$. There is no
solution $g \in \cC$ if the wealth-consumption ratio diverges at some
states, regardless of whether $\theta > 0$ or $\theta < 0$.\footnote{When
  $\theta < 0$, $g \to \infty$ implies $g = 0$ at some states, which means
  that $g \notin \cC$. Note that $W_t/C_t \geq 1$ by definition, so the
  wealth-consumption ratio becoming too low will not affect existence.} This
observation can help explain the existence condition in the applications
below, since the wealth-consumption ratio is directly affected by model
parameters.


\section{Decomposition of the Stability Exponent}\label{stab_coef}

In this section, we show that, under an independence condition, the stability
coefficient $\sS$ can be decomposed into three terms.  In stating the result, we
use the notation
    $\rR_a (Y) := \left(\EE Y^a \right)^{1/a}$
for any nonzero $a \in \RR$ and positive random variable $Y$.


\begin{proposition}\label{p:d}
    Let the conditions of Assumption~\ref{a:q} hold.  If $\{ C_t \}$ and $\{
    \lambda_t \}$ are independent, then the stability coefficient $\sS$ admits
    the alternative representation
    %
    \begin{equation}\label{eq:la}
        \sS = \ln \beta 
            + \sS_\lambda 
            + \left(1-\frac{1}{\psi}\right) \sS_c,
    \end{equation}
    where
    %
    \begin{equation}\label{eq:mm}
        \sS_\lambda
        := \lim_{T \to \infty}
        \frac{1}{T} \ln 
        \rR_\theta \left( \frac{\lambda_T}{\lambda_0} \right)
        \quad \text{and} \quad
        \sS_c
        := \lim_{T \to \infty}
        \frac{1}{T} \ln 
        \rR_{1-\gamma} \left( \frac{C_T}{C_0} \right).
    \end{equation}
    %
\end{proposition}

Proposition~\ref{p:d} separates the influence of the time preference parameter
$\beta$, the preference shocks $\{ \lambda_t \}$ and the consumption path $\{
C_t \}$ on existence and uniqueness of recursive utility. Note that if there are no
preference shocks, then $\lambda_t = 1$ for all $t$ and $\sS_\lambda = 0$. The proof of
Proposition~\ref{p:d} uses a local spectral radius theorem, which is employed
to obtain an alternative representation of the spectral radius $r(\KK)$ of
$\KK$.  (See the appendix for details.)

The independence condition required by Proposition~\ref{p:d} holds in many
(but not all) applications.\footnote{Exceptions include the extended model in
    \cite{Albuquerque2016} and the model of \cite{creal2020bond}.} The
    applications we consider in this paper all satisfy the condition.\footnote{The
        independence condition might seem problematic in our setup, given
        \eqref{eq:kappa}, which suggests that consumption growth and
        preference shocks depend on the same variables. However, in standard
        applications, it is commonly the case that $\{X_t\}$ and $\{\xi_t\}$
        are vector-valued with at least some independent components, and
        consumption and the preference shock are separated across these
    components (i.e., the components that drive the preference shock are
independent of the components that drive consumption).}

\begin{proposition}\label{p:ss}
    If the growth rate of $\{\lambda_t\}$ has zero mean, then $\sS_\lambda \cdot \theta  \geq 0$.
\end{proposition}

\begin{proof}
    In view of \eqref{eq:kappa}, we have
    %
    \begin{equation*}
        \sS_\lambda 
        = \lim_{T \to \infty} \frac{1}{T}
            \ln \rR_\theta \left( \frac{\lambda_T}{\lambda_0} \right)
        =  \frac{1}{\theta}
             \lim_{T \to \infty} \frac{1}{T}
             \ln \EE \exp \left( \theta \sum_{t=1}^T g_t \right),
    \end{equation*}
    % 
    where $g_t := g_\lambda(X_{t-1}, X_t, \xi_t)$.  Since $g_t$ has zero mean,
    so does $Z_T := \theta \sum_{t=1}^T g_t$.  As a result, $\ln \EE \exp
    (Z_T) \geq 0$.\footnote{If $Z$ satisfies $\EE Z = 0$,
        then, by Jensen's inequality, $0 = \EE \ln \exp(Z) \leq \ln \EE
    \exp(Z)$.}  It follows directly that $\sS_\lambda \cdot \theta  \geq 0$.
\end{proof}
%
The key implication of Proposition~\ref{p:ss} is that the sign of
$\sS_\lambda$ switches with $\theta$.  If $\theta \leq 0$, then $\sS_\lambda
\leq 0$, so adding preference shocks always lowers the stability coefficient
$\sS$ and loosens the existence condition.  If $\theta \geq 0$, then the
reverse is true.

Proposition~\ref{p:ss} matches observations previously made in some special
cases.  If $\theta \leq 0$, mean zero preference shocks lower the mean
wealth-consumption ratio  (see \cite{Albuquerque2016} and
\cite{schorfheide2018identifying}).  A lower wealth-consumption ratio also
implies that a solution is more likely in this case, as shown in \eqref{eq:wc}.

If $\theta \geq 0$, it follows that $\sS_\lambda \geq 0$, so adding
preference shocks tightens the existence condition. In this case, the income
effect dominates the substitution effect so adding preference shocks
increases the wealth-consumption ratio.\footnote{In terms of utility levels,
  an increase in the wealth-consumption ratio lowers utility when $\theta
  \geq 0$. As shown in \cite{Pohl2019}, the existence problem shows up at the
  lower end in this case.} Hence, adding preference shocks make the existence
condition more stringent. Furthermore, the more volatile and the larger the
persistence of the preference shock, the larger becomes $\sS$. We analyze the
quantitative impact of preference shocks on existence in Section \ref{s:app}.

Proposition \ref{p:ss} has valuable practical implications. For example, in
long-run risk models, standard parameterizations imply that $\theta<0$. Hence
$\sS_\lambda \leq 0$.  Thus, if a solution for a model without preference
shocks exists, there also exists a solution for the same model with preference
shocks. As a result, one can use the results in, say,
\cite{hansen2012recursive}, \cite{Pohl2019}, \cite{borovicka2020necessary}, or
\cite{christensen2022existence}, to prove existence for asset pricing models
without preference shocks. Under a zero mean growth rate specification,
existence for the same model with preference shocks then follows from
Proposition~\ref{p:ss}. 




\section{Applications}\label{s:app}

In this section we consider four applications.  All four assume that
$\{\lambda_t\}$ is independent of consumption and that the growth rate obeys
the AR1 specification
%
\begin{equation}\label{eq:lambda_ar1}
    \ln \left( \frac{\lambda_{t+1}}{\lambda_t} \right) 
    = h_{\lambda, t+1} 
    = \rho_\lambda h_{\lambda, t} + s_\lambda \, \eta_{\lambda, t+1} 
    \quad \text{where} \quad
    \{ \eta_{\lambda, t+1} \} \iidsim N(0,1).
\end{equation}
%
This allows us to derive an analytical expression for $\sS_\lambda$, through
which we sharpen and clarify the results stated above. 

The first two applications we consider are elementary: the benchmark model
with {\sc iid} consumption growth in \cite{Albuquerque2016} and the constant
volatility long-run risk model in \cite{bansal2004risks}.  In these settings,
we can obtain closed form expressions for $\sS_c$, and hence full analytical
expressions for $\sS$.  From these representations we can explore the
intuition behind the main stability condition and illustrate how volatility
affects stability under different parameterizations.

The second two applications are the model of \cite{schorfheide2018identifying}
and the model of \cite{GomezYaron2020}.  In these specifications, volatility
is time varying and an analytical expression for $\sS_c$ no longer exists.
In response, we explore numerical methods for analyzing stability.


\subsection{Purely Transient Consumption Growth}

We begin our analysis with the benchmark model of \cite{Albuquerque2016}.
In this application, log consumption growth is given by
%
\begin{equation}\label{eq:alcg}
	g_{c, t+1} 
	= \ln \left( \frac{C_{t+1}}{C_t} \right)    
	= \mu_c + \sigma_{c} \, \xi_{c, t+1},
\end{equation}
%
where $\{\xi_{c, t}\}$  is  {\sc iid} and standard normal.  The preference shock
process follows \eqref{eq:lambda_ar1}.  

\begin{proposition}\label{p:ar1}
	%
    Under the dynamics in \eqref{eq:lambda_ar1}--\eqref{eq:alcg}, we have
	%
	\begin{equation}\label{eq:ml_ar1}
        \sS_\lambda 
            =  \theta \, \frac{s_\lambda^2}{2(1-\rho_\lambda)^2}
            \quad \text{and} \quad
		\sS_c 
            = \mu_c + \frac{1}{2}(1-\gamma) \sigma_c^2,
	\end{equation}
	%
    so that
    %
    \begin{equation}\label{eq:iiss}
    \sS = \ln \beta 
            + \theta \, \frac{s_\lambda^2}{2(1-\rho_\lambda)^2}
            + \left(1-\frac{1}{\psi}\right) 
            \left[
                \mu_c + \frac{1}{2}(1-\gamma) \sigma_c^2
            \right].
    \end{equation}
    %
\end{proposition}


Notice how the expression for $\sS_\lambda$ in Proposition~\ref{p:ar1}
strengthens Proposition~\ref{p:ss} under the AR1 specification
\eqref{eq:lambda_ar1}.  We now have the strict inequality implication
$\sS_\lambda <0$ whenever $\theta<0$, as well as a direct connection between
$\sS_\lambda$ and the parameters in \eqref{eq:lambda_ar1}.  Increases in volatility $s_\lambda$
and persistence $\rho_\lambda$ of the preference shocks both lower the mean
wealth-consumption ratio and decrease $\sS_\lambda$, which makes the existence
of a solution more likely. 

When elasticity of intertemporal substitution is greater than one, the sign of
$\theta$ depends on risk aversion.  If the agent prefers early resolution of
risk (i.e., $\gamma > 1$), as is the case in most asset pricing applications,
then volatility in the preference shock suppresses the wealth-consumption ratio.  This
mitigates the possibility of divergence and hence promotes existence.
If the agent prefers late resolution of
risk, then the sign of $\theta$ is reversed and the opposite logic is true.

The closed-form expression for $\sS_\lambda$ can directly be related to the
singularity issues pointed out by \cite{deGroot2018}. Assuming $\gamma > 1$
as in standard parameterizations, it follows that $\sS_\lambda \to -\infty$
as $\psi \downarrow 1$ and $\sS_\lambda \to \infty$ as $\psi \uparrow 1$. In
both cases, preference shocks have a large impact on the wealth-consumption
ratio as well as utility. This is consistent with the strong effects of
preference shocks on equilibrium outcomes for IES close to one in
\cite{deGroot2018}. As a further illustration, we plot $\sS_\lambda$ as a
function of $\psi$ in the neighborhood of unity in Figure
\ref{f:singularity_albu}. Note that as $\psi \to 1$, $|\sS_\lambda|$
increases significantly, which resembles the asymptote in the responses to
preference shocks documented in \cite{deGroot2018}.

\begin{figure}
	\centering
	\includegraphics[width=0.6\textwidth]{figs/singularity_albu.pdf}
    \caption{The figure plots $\sS_\lambda$ as a function of $\psi$ in the
    neighborhood of unity for the benchmark model of \cite{Albuquerque2016}.}\label{f:singularity_albu}
\end{figure}


The analytic solution for $\sS_c$ in \eqref{eq:ml_ar1} is derived in
\cite{borovicka2020necessary}.  Note that for $\psi>1$  the substitution
effect dominates the wealth effect so the investor lowers her consumption
relative to wealth in response to improved investment opportunities. So for a
given $\mu_c > 0$, either $\beta$ must be sufficiently small, $\sigma_{c}$
must be sufficiently large or $h_{\lambda, t+1}$ must be sufficiently
persistent and volatile to guarantee existence.  \cite{Albuquerque2016} use
$\beta = 0.99795$, $\gamma = 1.516$, $\psi =
1.4567$, $\mu_c = 0.0015644$, $\sigma_c = 0.0069004$, $\rho_\lambda = 0.99132$
  and $s_\lambda = 0.00058631$, which implies
%
\begin{equation*}
	\ln \beta  = -0.00205,
    \quad
	\sS_\lambda  =  -0.00375,
    \quad \text{and} \quad
	\left(1-\frac{1}{\psi}\right) \sS_c = 0.00049.
\end{equation*}
%
Hence $\sS = -0.0053$.  From Theorem~\ref{t:bk} it follows that the model has a unique stationary
Markov solution in $\cC$.  Furthermore, the effect of the preference shock on
$\sS$ is sufficiently large such that for any  $\beta< 1$ it holds that $\sS
< 0$.  Figure \ref{f:stab_coeff_lambda_albu_2} plots  $\sS_\lambda$  as a
function of $\rho_\lambda $ and $s_\lambda$. The more persistent and the more
volatile the growth rate of the preference shock, the smaller  $\sS_\lambda$.
The magnitudes of $\sS_\lambda$ are large compared to
$\left(1-1/\psi\right) \sS_c$, so adding a preference shock to the
model makes existence of a solution significantly more likely. 

\begin{figure}
	\centering
	\includegraphics[width=0.7\textwidth]{figs/stab_coeff_lambda_albu2.pdf}
    \caption{The figure plots $\sS_\lambda$ as a function of $\rho_\lambda $
    and $s_\lambda$ for the benchmark model of
    \cite{Albuquerque2016}.}\label{f:stab_coeff_lambda_albu_2}
\end{figure}

The size of this effect is demonstrated in Figure
\ref{f:stab_coeff_comparison_3d_albu}. It plots the stability coefficient
$\sS$ as a function of $\psi $ and $\gamma$ with (the darker surface)
and without (the lighter surface) preference shocks. We observe that $\sS$ is
significantly lower in the benchmark model and increasing $\gamma$ or
decreasing $\psi$ has a large impact on $\sS$ when preference shocks are
present, while the impact is significantly smaller in the model without
preference shocks.

\begin{figure}
    \centering
	\includegraphics[width=0.65\textwidth, trim=0 18mm 0 20mm, clip]{figs/stab_coeff_comparison_3d_albu}
	\caption{\label{f:stab_coeff_comparison_3d_albu} The figure plots $\sS$
      as a function of $\gamma$ and $\psi$ for the model of
      \cite{Albuquerque2016}. The darker surface on the bottom and
      the lighter surface above show the results with and without preference
      shocks, respectively.}
\end{figure}

\subsection{Constant Volatility Consumption Growth}

Following Section~I.A of \cite{bansal2004risks}, 
suppose that consumption grows according to
%
\begin{align*}
    g_{c, t+1} 
    & = \mu_c + z_t + \sigma_c \, \xi_{t+1}
        \\
    z_{t+1}
    & = \rho z_t + \sigma \, \eta _{t+1}.
\end{align*}
%
The specification for preference shocks remains as in \eqref{eq:lambda_ar1}.
In this case, analysis similar to that for Proposition~\ref{p:ar1} yields
%
\begin{equation}\label{eq:sscv}
    \sS 
    = \ln \beta
    + \theta \, \frac{s_\lambda^2}{2(1-\rho_\lambda)^2}
    +
    \left(1-\frac{1}{\psi}\right)\left[\mu_c + \frac{1}{2}(1-\gamma)
    \left( \sigma_c^2 + \frac{\sigma^2}{(1-\rho)^2} \right)\right].
\end{equation}
%
Compared to $\sS$ in \eqref{eq:iiss}, the only difference is in the final term
$\sigma^2 / (1-\rho)^2$.  This term arises from the persistent, constant
volatility, long-run growth component $\{z_t\}$.

We see that, in the risk-averse, high elasticity of intertemporal substitution
setting, where $\gamma > 1$ and $\psi > 1$, the existence condition tightens
when the mean growth rate for consumption is high and the persistence and volatility of
consumption is low.  Conversely, for this risk-averse agent, high volatility and high persistence in
consumption growth relaxes the existence condition by reducing lifetime
utility.



\subsection{The Long-Run Risk Model of \cite{schorfheide2018identifying}}

In this section we investigate the stability properties of the long-run risk
model of \cite{schorfheide2018identifying}. In the model, consumption growth obeys
%
\begin{equation}\label{eq:ssycg}
    g_{c, t+1} 
    = \ln \left( \frac{C_{t+1}}{C_t} \right)    
       = \mu_c + z_t + \sigma_{c, t} \, \xi_{c, t+1},
\end{equation}
%
where
%
\begin{equation*}
    z_{t+1} 
        = \rho \, z_t + \sqrt{1 - \rho^2} \, 
                \sigma_{z, t} \, \eta_{t+1}
\end{equation*}
%
and
%
\begin{equation*}
    \sigma_{i,t} 
        = \phi_i \, \bar{\sigma} \, \exp(h_{i, t})
        \quad \text{where} \quad
    h_{i, t+1}
        = \rho_i \, h_{i,t} + s_i \, \eta_{i, t+1}
        \quad \text{for } i \in \{z, c\}.
\end{equation*}
%
The process $\{\lambda_t\}$ is unchanged from \eqref{eq:lambda_ar1}. All
innovations $(\xi_{c, t}, \eta_t,  \eta_{z, t}, \eta_{c, t}, \eta_{\lambda,
t})$ are {\sc iid} and standard normal.  The state vector $x$ contains four
states and is given by
%
\begin{equation}\label{eq:state}
    x = (z, h_z, h_c, h_\lambda) \in \XX := \RR^4.
\end{equation}

In this setting, the operator $\KK$ takes the form
%
\begin{equation*}
    (\KK g)(x)
    = \EE_x
        \, g(X_{t+1}) 
        \exp \left\{
            \theta h_{\lambda, t+1}
            + (1-\gamma)(\mu_c + z + \sigma_{c, t} \xi_{c, t+1})
        \right\}.
\end{equation*}
%
We can reduce dimensionality in the conditional expectation by integrating out
the independent innovation $\xi_{c, t+1}$, which leads to 
%
\begin{align*}
    (\KK g)(x)
    & = \EE_x \, g(X_{t+1}) 
        \exp \left\{
            \theta h_{\lambda, t+1}
            + (1-\gamma)(\mu_c + z)
            + \frac{1}{2} (1-\gamma)^2 \sigma_{c, t}^2 
        \right\}
        \\
    & = \exp \left\{
            (1-\gamma)(\mu_c + z)
            + \frac{1}{2} (1-\gamma)^2 \sigma_{c, t}^2 
        \right\}
        \EE_x \, g(X_{t+1}) 
        \exp \left\{
            \theta h_{\lambda, t+1}
        \right\},
\end{align*}
%
where the conditioning is on $X_t = x$ as given in \eqref{eq:state}. The
consumption dynamics \eqref{eq:ssycg} do not permit an analytic solution for
$\sS_c$. Hence, we compute the stability coefficient using numerical methods.
As proposed by \cite{borovicka2020necessary}, we compute $\sS_c$ using Monte
Carlo simulations (see Appendix~\ref{ss:N_T} for details on the computations).  \cite{schorfheide2018identifying} use $\beta = 0.999$, $\gamma = 8.89$, $\psi
=
1.97$, $\mu_c = 0.0016$, $\bar{\sigma} = 0.0035$, $\phi_z = 0.215 \sqrt{1 -
\rho^2}$, $\phi_c = 1$, $\rho = 0.987$, $\rho_\lambda = 0.959$, $\rho_z =
0.992$, $\rho_c = 0.991$, $s_\lambda = 0.0004$, $s_z^2 = 0.0039$ and $s_c^2 =
0.0096$, which implies
%
\begin{equation*}
    \ln \beta = -0.001,\; \sS_\lambda = -0.00076, \text{ and }
    \left(1-\frac{1}{\psi}\right) \sS_c = 0.00061.
\end{equation*}
%
Hence, $\sS = -0.00115$. From Theorem~\ref{t:bk} it follows that the model has
a unique stationary Markov solution in $\cC$. Again, we find that the effect
of the preference shock on $\sS$ is sufficiently large such that for any
$\beta < 1$ it holds that $\sS < 0$.  


%\begin{figure}
%    \centering
%    \scalebox{0.6}{\includegraphics{figs/stab_coeff_lambda_ssy_1.pdf}}
%    \caption{\label{f:stab_coeff_lambda_ssy_1} $\sS_\lambda$ as a function of $\psi, \gamma$ (SSY model)}
%\end{figure}

\begin{figure}
	\centering
	\includegraphics[width=0.7\textwidth]{figs/stab_coeff_lambda_ssy_2.pdf}
	\caption{\label{f:stab_coeff_lambda_ssy_2} The figure plots $\sS_\lambda$ as a function of $\rho_\lambda $
		and $s_\lambda$ for the model of \cite{schorfheide2018identifying}.}
\end{figure}

Figure~\ref{f:stab_coeff_lambda_ssy_2}
plots $s_\lambda$ as a function of $\rho_\lambda$ and $s_\lambda$.
It shows that even small changes in $\rho_\lambda$ and $s_\lambda$ have large
effects on existence. While highly persistent and volatile preference shocks
guarantee existence even if $\left(1-\frac{1}{\psi}\right) \sS_c$ is large,
slightly lower values for $\rho_\lambda$ and $s_\lambda$ compared to the
calibration of  \cite{schorfheide2018identifying} imply a negligible effect of
the preference shocks on existence. For example, assuming  $\rho_\lambda =
0.94$ and $s_\lambda = 0.0036$ instead of  $\rho_\lambda = 0.959$ and
$s_\lambda = 0.0004$ implies $s_\lambda = -0.00028$: the effect of
preference shocks on existence is almost reduced by a factor of 3. 

Figure \ref{f:stab_coeff_comparison_3d_ssy} plots $\sS$ as a function of
$\gamma$ and $\psi$. For the model with preference shocks (the darker
surface), we find that increasing risk aversion and decreasing the IES towards
unity significantly lowers $\sS$ so it makes the existence condition less
tight. In contrast, in the absence of preference shocks (the lighter
surface), the effect of risk aversion and the IES on existence is much
smaller. Furthermore, adding preference shocks to the model significantly
relaxes the existence condition, which can be observed by the large absolute
difference between the two surfaces. The dashed line shows the difference in
$\sS$ between the model with and without preference shocks.

 \begin{figure}
	\centering
	\includegraphics[width=0.6\textwidth, trim=0 18mm 0 20mm, clip]{figs/stab_coeff_comparison_3d_ssy}
	\caption{\label{f:stab_coeff_comparison_3d_ssy}The figure plots $\sS$ as
      a function of $\gamma$ and $\psi$ for the model of
      \cite{schorfheide2018identifying}. The darker surface on the bottom and
      the lighter surface above show the results with and without preference
      shocks, respectively.}
\end{figure}

%\begin{figure}
%	\centering
%	\includegraphics[width=\textwidth]{figs/stab_coeff_ssy_2}
%	\caption{\label{f:stab_coeff_ssy_2} The figure plots $\sS$ as a function of $\rho_\lambda $
%		and $s_\lambda$ for the model of \cite{schorfheide2018identifying}.}
%\end{figure}
%\begin{figure}
%    \centering
%    \scalebox{0.6}{\includegraphics{figs/singularity.pdf}}
%    \caption{\label{f:singularity} $\sS_\lambda$ as a function of $\psi$ in the neighborhood of unity}
%\end{figure}
%\begin{figure}
%	\centering
%	\begin{subfigure}{.5\textwidth}
%		\caption{Stability coefficient $\sS$}
%		\includegraphics[width=\textwidth]{figs/stab_coeff_ssy_1}
%	\end{subfigure}%
%	\begin{subfigure}{.5\textwidth}
%		\caption{Stability coefficient $\sS$ with $\sS_\lambda = 0$}
%		\includegraphics[width=\textwidth]{figs/stab_coeff_nolambda_ssy.pdf}
%	\end{subfigure}
%    \caption{The figure plots $\sS$ as a function of $\psi $ and $\gamma$ for
%      the benchmark model of \cite{schorfheide2018identifying}. Panel (A)
%      shows results for the model with a preference shock, panel (B) shows
%      the results without a preference shock so that $\sS_\lambda = 0$.}
%\end{figure}
% \begin{figure}
% 	\centering
% 	\scalebox{0.7}{\includegraphics[trim=20mm 20mm 1mm 20mm, clip]{figs/ssy_wcr_zhz.pdf}}  % left bottom right top
% 	\caption{\label{f:ssy_wcr_zhz} The wealth-consumption ratio given $z$ and $h_z$ (SSY model)}
% \end{figure}


\subsection{The Long-Run Risk Model of \cite{GomezYaron2020}}


In this section, we analyze the stability properties of the model of \cite{GomezYaron2020}. The authors add inflation dynamics to a long-run risk model similar to the one of 
\cite{schorfheide2018identifying}. In particular, they assume that the expected inflation rate $z_{\pi,t}$ affects the mean growth rate of consumption:
%
\begin{equation}\label{eq:gcycg}
	g_{c, t+1} 
	= \ln \left( \frac{C_{t+1}}{C_t} \right)    
	= \mu_c + z_t + \sigma_{c, t} \, \xi_{c, t+1},
\end{equation}
%
where
%
\begin{align*}
	z_{t+1} 
	&= \rho \, z_t + \rho_{\pi} \, z_{\pi,t}+ 
	\sigma_{z, t} \, \eta_{t+1}\\
	z_{\pi,t+1} 
	&= \rho_{\pi \pi} \, z_{\pi,t} + \sigma_{z \pi, t} \, \eta_{\pi,t+1}
\end{align*}
%
and
%
\begin{equation*}
	\sigma_{i,t} 
	= \phi_i \, \bar{\sigma} \, \exp(h_{i, t})
	\quad \text{where} \quad
	h_{i, t+1}
	= \rho_i \, h_{i,t} + s_i \, \eta_{i, t+1}
	\quad \text{for } i \in \{z, c, z\pi \}.
\end{equation*}
%
Note that also the expected inflation rate $z_{\pi,t}$ has stochastic volatility $\sigma_{z\pi,t}$. As in the model of \cite{schorfheide2018identifying}, the process $\{\lambda_t\}$ follows
\eqref{eq:lambda_ar1} and all shocks are \textsc{iid} and standard normal. Hence, the state vector $x$ contains 6 states and is given by
%
\begin{equation*}
    x = (z, z_\pi, h_z, h_c, h_{z\pi}, h_\lambda) \in \XX := \RR^6. 
\end{equation*}
%
We can apply the same conditioning as in the model of
\cite{schorfheide2018identifying} and we need to compute $\sS_c$ numerically
again. \cite{GomezYaron2020} use $\beta = 0.9987$, $\gamma = 13.01$, $\psi =
1.5$, $\mu_c = 0.0016$, $\bar{\sigma} = 0.0015$, $\phi_z = 0.13$, $\phi_c =
1$, $\phi_{z\pi} = 0.08$, $\rho = 0.983$, $\rho_{\pi} = -0.0075$,
$\rho_{\pi\pi} = 0.985$, $\rho_\lambda = 0.981$, $\rho_z = 0.980$, $\rho_c =
0.992$, $\rho_{z\pi} = 0.970$, $s_\lambda = 0.00018$, $s_z = 0.09$, $s_c =
0.104$ and $s_{z\pi} = 0.271$, which implies
%
\begin{equation*}
\ln \beta = -0.0013,\; \sS_\lambda = -0.0016, \text{ and }
\left(1-\frac{1}{\psi}\right) \sS_c = 0.0004.
\end{equation*}
%
Hence, $\sS = -0.0025$. From Theorem~\ref{t:bk} it follows that the model has a unique stationary
Markov solution in $\cC$. As in the other models considered in this section, the effect of the preference shock on $\sS$ is sufficiently large such that for any $\beta < 1$ it holds that $\sS < 0$. Also, as Figure~\ref{f:stab_coeff_lambda_gcy_2} shows, even small changes in the persistence and volatility of the preference shock have large effects on existence. So slightly increasing $\rho_\lambda $ and $s_\lambda$ will ensure existence even in a rapidly growing economy.

Figure~\ref{f:stab_coeff_comparison_3d_gcy} plots $\sS$ as a function of
$\gamma$ and $\psi$ for the model with and without preference shocks. As in
the model of \cite{schorfheide2018identifying}, $\sS$ is significantly lower
in the model with preference shocks (the darker surface) compared to the same
model without preference shocks (the lighter surface) and the difference
increases rapidly with increasing risk aversion and decreasing IES towards
unity. Hence, in the three quantitative applications we consider, preference
shocks are the dominant driver for the existence of a solution. The shocks
ensure existence for any discount factor $\beta < 1$ and solutions exists
even if the mean growth rate is high.
\begin{figure}
	\centering
	\includegraphics[width=0.7\textwidth]{figs/stab_coeff_lambda_gcy_2.pdf}
	\caption{\label{f:stab_coeff_lambda_gcy_2} The figure plots $\sS_\lambda$ as a function of $\rho_\lambda $
		and $s_\lambda$ for the model of \cite{GomezYaron2020}.}
\end{figure}
\begin{figure}
	\centering
	\includegraphics[width=0.6\textwidth, trim=0 18mm 0 20mm, clip]{figs/stab_coeff_comparison_3d_gcy}
	\caption{\label{f:stab_coeff_comparison_3d_gcy}The figure plot $\sS$ as a
      function of $\gamma$ and $\psi$ for the model of \cite{GomezYaron2020}.
      The darker surface on the bottom and the lighter surface above show the
      results with and without preference shocks, respectively.}
\end{figure}


%\begin{figure}
%    \centering
%    \includegraphics[width=0.7\textwidth]{figs/stab_coeff_gcy_2}
%    \caption{\label{f:stab_coeff_gcy_2} $\sS$ as a function of $\rho_\lambda$
%      and $s_\lambda$ (GCY model)}
%\end{figure}

%\begin{figure}
%	\centering
%	\begin{subfigure}{.5\textwidth}
%		\caption{Stability coefficient $\sS$}
%		\includegraphics[width=\textwidth]{figs/stab_coeff_gcy_1}
%	\end{subfigure}%
%	\begin{subfigure}{.5\textwidth}
%		\caption{Stability coefficient $\sS$ with $\sS_\lambda = 0$}
%		\includegraphics[width=\textwidth]{figs/stab_coeff_nolambda_gcy.pdf}
%	\end{subfigure}
%    \caption{The figure plots $\sS$ as a function of $\psi $ and
%    $\gamma$ for the benchmark model of \cite{GomezYaron2020}. Panel (A)
%    shows results for the model with a preference shock, panel (B) shows the
%    results without a preference shock so that  $\sS_\lambda = 0$.}
%\end{figure}



\section{An Alternative Specification}\label{s:alt_spec}

To remove the asymptote in the responses to preference shocks,
\cite{deGroot2018} propose the following alternative specification for
recursive preferences
%
\begin{equation}
    \label{eq:agg_alt}
    V_t = \left[
            (1 -  a_t\beta) C_t^{1-1/\psi}
            + a_t\beta \left\{ \rR_{t, 1-\gamma} \left(V_{t+1}
            \right) \right\}^{1-1/\psi}
          \right]^{1/(1-1/\psi)},
\end{equation}
%
which \cite{degroot2021valuation} then use in an asset pricing context.

Dividing both sides of \eqref{eq:agg_alt} by $C_t$ and raising to the power
of $1 - \gamma$ yields an equivalent recursion
%
\begin{equation}
    \label{eq:agg2_alt}
    \tilde G_t = \left\{
        (1 - a_t\beta) + a_t\beta
        \left[
            \EE_t \tilde G_{t+1} \left(\frac{C_{t+1}}{C_t}\right)^{1-\gamma}
        \right]^{1/\theta}
        \right\}^\theta,
\end{equation}
%
where
%
\begin{equation*}
    % \label{eq:defg_alt}
    \tilde G_t := \left( \frac{V_t}{C_t} \right)^{1-\gamma}.
\end{equation*}
%
Similar to Section~\ref{s:results}, we seek a stationary Markov solution to
\eqref{eq:agg2_alt} in the form of $\{\tilde G_t\} := \{\tilde g(X_t)\}$. Let
operator $\tilde \KK$ be defined by
%
\begin{equation*}
    % \label{eq:defk_alt}
    (\tilde \KK g)(x) = \EE_x g(X_{t+1})
    \exp\left\{(1-\gamma) g_c(X_t, X_{t+1}, \xi_{t+1}) \right\}
\end{equation*}
%
and assume that $a_t = h(X_t) \in (0, 1/\beta)$. Let $\tilde F: \RR_+ \times
\XX \to \RR$ be given by
%
\begin{equation*}
    \tilde F(t, x) = \left(1 - h(x)\beta + h(x)\beta t^{1/\theta} \right)^\theta.
\end{equation*}
%
Then, the solution $\tilde g$ satisfies
%
\begin{equation}
    \label{eq:deft_alt}
    \tilde g(x)
    = (\tilde \TT \tilde g) (x)
    := \tilde F \left( (\tilde \KK \tilde g)(x), x \right).
\end{equation}
%
We have the following proposition that parallels Theorem~\ref{t:bk}.

\begin{proposition}
    \label{p:drt}
    Suppose $h \in \cC$ and $\sup_{x \in \XX} h(x) = \bar a < 1/\beta$. If
    Assumption~\ref{a:q} holds and
    %
    \begin{equation}
        \label{eq:sS_alt}
        \tilde \sS := \ln \beta + \ln \bar a + \frac{\ln r(\tilde \KK)}{\theta}
        < 0,
    \end{equation}
    %
    then the recursion \eqref{eq:agg2_alt} has a unique stationary Markov
    solution $\tilde g^*$ in $\cC$. In addition, for all $g \in \cC$, there
    exists an $a < 1$ and $N < \infty$ such that
    %
    \begin{equation*}
        \sup_{x \in \XX} | (\tilde \TT^n g)(x) - \tilde g^*(x) |
        \leq a^n N
        \;
        \text{ for all } n \in \NN.
    \end{equation*}
    %
\end{proposition}

The main difference between condition \eqref{eq:sS_alt} and $\sS < 0$ in
Theorem~\ref{p:drt} is that $\tilde \sS$ does not contain the growth rate of
preference shocks. Instead, only the maximum level of the preference shock
process affects existence. In this case, no matter whether $\psi$ approaches 1 from
above or below, $\lim_{\psi \to 1}\tilde \sS = \ln \beta + \ln \bar a$. This
confirms the lack of asymptote reported in \cite{deGroot2018} and \cite{degroot2021valuation}.

\section{Conclusion}\label{s:conclusion}

%\textcolor{red}{I think we might want to talk about the big picture, the
  %bigger theme, in the first few sentences here. That is, we contribute to
  %the understanding of preference shocks in asset pricing models and that's
  %important. (I don't think we can claim to have resolved the debates on
  %preference shocks. Those papers in Econometrica were debating whether we
  %need certain kind of preference shocks to generate large responses, but we
  %don't answer this question in this paper.) And in the end, we could discuss
  %directions for future research, for example, how to connect our results to
  %the economy's response to shocks.}

The results presented in this paper substantially enhance our understanding of
lifetime utility and the wealth-consumption ratio in modern asset pricing
models with preference shocks.  The core results in our paper are built around
a stability coefficient that depends on preference parameters, preference
shock dynamics, and the dynamics of consumption paths.
To isolate the effect of preference shocks on existence, we use a weak
independence assumption to show that the stability coefficient can be
decomposed into three terms: the first term only depends on the subjective
discount factor and the second and third terms are driven by preference shocks
and consumption risks respectively. This decomposition allows us to quantify
the impact of preference shocks on existence. 

We show that, for risk averse agent with an IES greater than one---as is
commonly assumed in asset pricing models---preference shocks always relax the
existence condition and, moreover, that their impact on existence is large under
standard calibrations.  More precisely, preference
shocks dominate the effect of consumption risks on existence, so that, even for
quickly growing economies, there exists a solution for any subjective discount
factor smaller than one.  Moreover, if the IES approaches one, the stability
coefficient diverges such that for an IES sufficiently close but larger
(smaller) than one, a solution always (never) exists. In sum, this paper
provides a general and easily applicable approach that can be used to settle
the questions of existence and uniqueness for a wide range of asset pricing
models with preference shocks.

As our approach does not require strong assumptions on the underlying
consumption process, it can be extended to cases in which preference shocks
are added to models that rely on other risk factors such as models with
consumption disasters as in \cite{Barro2009} and \cite{Wachter2013}, models
with volatility of volatility as in \cite{Bollerslev2009} and
\cite{Bollerslevetal2015} and models with jumps in volatility and growth rates
as in \cite{DrechslerYaron11}.  The details of these extensions and the impact
of other risk factors on the stability coefficient are left for future
research.





\section{Appendix}

In this section we detail computations and collect remaining proofs.
We also provide a variation on a fixed point theorem of \cite{du1990fixed} that has
independent interest.

\subsection{Preliminaries}


Let $\eE := (\eE, \| \cdot \|, \leq)$ be a Banach lattice (see, e.g,
\cite{meyer2012banach}).  A self-map $S$ from a convex subset $E$ of $\eE$
into itself is called convex on $E$ if 
%
\begin{equation*}
    S (\lambda f + (1-\lambda) g) \leq \lambda S + (1-\lambda) S g
    \text{ for all }
    f,g \in E
    \text{ and all }
    \lambda \in [0,1].   
\end{equation*}
%
It is called concave if the reverse inequality holds. It is called order
preserving if $f \leq g$ implies $Sf \leq Sg$.

For a linear operator $A$ mapping $\eE$ to itself, the operator
norm and spectral radius of $A$  are defined, as usual, by $\|A\| := \sup
\setntn{\|Ag\|}{g \in \eE ,\; \| g \| \leq 1}$ and $r(A) := \lim_{n \to
\infty} \| A^n \|^{1/n}$ respectively.   The operator $A$ is called
positive if $Ag \geq 0$ whenever $g \geq 0$.  It is called
bounded if $\| A \|$ is finite and compact if the image of the
unit ball in $\eE$ under $A$ has compact closure.

For a self-map $S$ from $E \subset \eE$ to itself, we will say that $S$ is
geometrically stable on $E$ if $S$ has a unique fixed point $u^*$ in $E$ and,
moreover, for all $u \in E$, there exists an $a < 1$ and $N < \infty$ such
that $\| S^n u - u^* \| \leq a^n N$ for all $n \in \NN$.  Let $\pP$ be the
(nonempty) interior of the positive cone of $\eE$. For $u, v \in \eE$, we
write $u \ll v$ if $v - u \in \pP$. The following fixed point result, which
extends a deeper result due to \cite{du1990fixed}, is the key to our main
theorem. 

\begin{theorem}\label{t:du}
    Let $S$ be a self-map on $\pP$ such that the following two conditions hold.
    %
    \begin{enumerate}
        \item For any $g \in \pP$, there exists an $f \in \pP$ such that $f \leq g$ and $f \ll Sf$.
        \item For any $g \in \pP$, there exists an $f \in \pP$ such that $g \leq f$ and $Sf \ll f$.
    \end{enumerate}
    %
    If, in addition, $S$ is order preserving and either convex or concave on
    $\pP$, then $S$ is geometrically stable on $\pP$.
\end{theorem}

\begin{proof}
    Fix $g_1 \in \pP$ and use (a) to take $f_1$ such that $f_1 \leq g_1$ and
    $f_1 \ll Sf_1$.  In addition, fix $g_2 \in
    \pP$ and use (b) to take $f_2$ such that $g_2 \leq f_2$ and $Sf_2 \ll f_2$. Since $S$ is order preserving, $S$ is a self-map on $[f_1,
    f_2]$.  Since $S$ is either convex or concave and the strict orderings
    $f_1 \ll Sf_1$ and $Sf_2 \ll f_2$ hold, $S$ is geometrically stable on
    $[f_1, f_2]$ by Theorem~3.1 of \cite{du1990fixed}.  Let $g^*$ be the fixed
    point.

    Pick any $g \in \pP$.  By the conditions of Theorem \ref{t:du}, we can now take a new
    pair of functions $h_1, h_2$, this time with $h_1 \leq g, g^* \leq h_2$,
    such that $Sh_1 \gg h_1$ and $Sh_2 \ll h_2$.  Applying Theorem~3.1
    from \cite{du1990fixed} once more, $S$ is geometrically stable on $[h_1,
    h_2]$.  Since $g^*$ is a fixed point of $S$ and $g^* \in [h_1, h_2]$, it
    must be that $g^*$ is the unique fixed point in $[h_1, h_2]$, and, for
    some $a <1$ and $N < \infty$, that $\| S^n g - g^*\| \leq a^n N$ for all $n \in \NN$.

    We have now shown that every $g \in \pP$ converges geometrically to the
    $g^*$, where $g^*$ is a fixed point of $S$.  As a consequence, $g^*$ is the unique fixed point of
    $S$ in $\pP$, and $S$ is geometrically stable on $\pP$.
\end{proof}





\subsection{Properties of The Discount Operator}

In this section we investigate the properties of $\KK$.  We adopt the setting of
Section~\ref{s:env} and all assumptions stated there are maintained here,
including Assumption~\ref{a:q}. 

Let $L_1(\pi)$ be all Borel measurable functions $g \colon \XX \to \RR$ with
%
\begin{equation*}
    \| g \| := \int |g(x)| \pi(\diff x) < \infty.
\end{equation*}
%
For $f, g \in L_1(\pi)$ we write $f \leq g$
if $f(x) \leq g(x)$ for $\pi$-almost all $x \in \XX$.  We write $f \ll g$ if
$f(x) < g(x)$ for $\pi$-almost all $x \in \XX$.  We define $\gG$ to be all $f
\in L_1(\pi)$ such that $f \gg 0$.

The next lemma will be useful.

\begin{lemma}
    The common marginal density $\pi$ of each $X_t$ is continuous and
    everywhere positive on $L_1(\pi)$.
\end{lemma}

\begin{proof}
    This follows from Assumption~\ref{a:q}.  The proof is identical to that of
    Lemma~C1 in \cite{borovicka2020necessary}.
\end{proof}

We will use a kind of local spectral radius theorem.  The version below is
proved in \cite{borovicka2020necessary}.

\begin{theorem}
    \label{t:lsr2}
    Let $A$ be a linear operator on $L_1(\pi)$.  If $A^i$ is compact for some
    $i \in \NN$ and $A g \in \gG$ whenever $g \in \gG$, then
    %
    \begin{equation}
        \label{eq:lsr2}
        \lim_{n \to \infty}
        \left\{ \int A^n h \, \diff \pi \right\}^{1/n} = r(A)
        \quad \text{for all } h \in \gG.
    \end{equation}
    %
\end{theorem}

Below we use Theorem~\ref{t:lsr2} to generate the alternative representation
of $\sS$ provided in Proposition~\ref{p:d}.

Throughout this section, we regard $\KK$ from \eqref{eq:defk} as a linear
operator on $L_1(\pi)$.  The spectral radius $r(\KK)$ of $\KK$ is the
$L_1(\pi)$ spectral radius.  We find it convenient to express $\KK$ as the
integral operator
%
\begin{equation}\label{eq:defkio}
    (\KK g)(x) = \int g(y) k(x, y) \diff y 
\end{equation}
%
where the kernel $k$ is given by 
%
\begin{equation*}
    k(x, y) 
    := \EE_\xi \Gamma(x, y, \xi)
    q(x, y) 
    \qquad ((x, y) \in \XX \times \XX)
\end{equation*}
%
with $\Gamma$ defined in \eqref{eq:sdef}.
Since $\EE_\xi\Gamma$ and $q$ are both continuous and $\XX$ is compact, the function
$k$ is continuous and bounded on $\XX \times \XX$.

\begin{lemma}
    \label{l:wpok}
    Regarding the operator $\KK$, the following statements are true:
    %
    \begin{enumerate}
        \item $\KK$ is a bounded linear operator on $L_1(\pi)$.
        \item $\KK g$ is continuous at each $g \in \gG$.
        \item $\KK g \geq 0$ when $g \geq 0$ and $\KK g \in \gG$ whenever $g \in \gG$.
        \item $\KK$ is irreducible and $\KK^2$ is compact.
        \item $r(\KK) > 0$ and there exists a continuous function $e \in \gG$
            satisfying $\KK e = r(\KK) e$.  
        \item $\KK$ is order preserving on $L_1(\pi)$.
    \end{enumerate}
    %
\end{lemma}

\begin{proof}[Proof of Lemma~\ref{l:wpok}]
    Proofs for (a)--(b) can be found in the proof of Lemma~C2 of
    \cite{borovicka2020necessary}.  (While the definition of the kernel $k$
    for the integral operator $\KK$ is
    different in \cite{borovicka2020necessary}, its properties are essentially
    identically.  As a result, no modifications to the proof are necessary.) 
    For part (c), the first claim is obvious and the second follows from
    everywhere positivity of $\Gamma$.
    Part (d) follows from Lemma~C3 of \cite{borovicka2020necessary}.

    Regarding part (e), Theorem~4.1.4 and Lemma~4.2.9 of
    \cite{meyer2012banach} together with the irreducibility and compactness
    properties of $\KK$ obtained in part (d) yield positivity of
    $r(\KK)$ and existence of the positive eigenfunction $e$.
    Part (b) implies that $e$ is continuous, since $e \in \gG$,
    $r(\KK)>0$ and $e = (\KK e) / r(\KK)$.

    Regarding part (f), we have already shown that $\KK$ is linear and $\KK f
    \geq 0$ when $f \geq 0$.  Hence, if $g \leq h$, then $\KK (h - g) \geq 0$
    and, therefore, $\KK h - \KK g \geq 0$.  
\end{proof}

In what follows, the function $e$ is called the Perron--Frobenius
eigenfunction.



\begin{lemma}\label{l:kn}
    For all $n \in \NN$, we have
    %
    \begin{equation}\label{eq:knar}
        (\KK^n \1)(x)
        = \EE_x
         \left( \frac{\lambda_n}{\lambda_0} \right)^\theta
         \left( \frac{C_n}{C_0} \right)^{1-\gamma} 
         \qquad (x \in \XX).
    \end{equation}
    %
\end{lemma}

\begin{proof}
    Fix $n \in \NN$.  A straightforward inductive argument confirms that
    %
    \begin{equation}\label{eq:kir}
        (\KK^n \1)(x) = \EE_x \prod_{i=1}^n \Gamma(X_{i-1}, X_i, \xi_i)
    \end{equation}
    %
    for all $x \in \XX$, where $\EE_x$ conditions on $X_0 = x$.  Now observe that
    %
    \begin{align*}
        \prod_{i=1}^n \Gamma(X_{i-1}, X_i, \xi_i)
        & =
        \prod_{i=1}^n
        \exp
        \left\{ 
            \theta g_\lambda(X_{i-1}, X_i, \xi_i) + (1-\gamma)
            g_c(X_{i-1}, X_i, \xi_i)
        \right\}
        \\
        & =
        \prod_{i=1}^n
        \left( \frac{\lambda_i}{\lambda_{i-1}} \right)^\theta
        \left( \frac{C_i}{C_{i-1}} \right)^{1-\gamma} .
    \end{align*}
    %
    Cancelling terms and combining with \eqref{eq:kir} gives \eqref{eq:knar}.
\end{proof}


\begin{proof}[Proof of Proposition~\ref{p:d}]
    From Lemma~\ref{l:kn} and the law of iterated expectations we have
    %
    \begin{equation*}
       \int \KK^n \1 \diff \pi
        = \EE
         \left( \frac{\lambda_n}{\lambda_0} \right)^\theta
         \left( \frac{C_n}{C_0} \right)^{1-\gamma} ,
    \end{equation*}
    %
    where $\EE$ is the unconditional stationary expectation (i.e., with $X_0 \eqdist \pi$).
    By independence of $\{\lambda_t\}$ and $\{ C_t \}$, we then have
    %
    \begin{align*}
       \left\{
           \int \KK^n \1 \diff \pi
       \right\}^{1/n}
       & =
        \left\{
            \EE \left( \frac{\lambda_n}{\lambda_0} \right)^\theta
        \right\}^{1/n}
        \left\{
            \EE \left( \frac{C_n}{C_0} \right)^{1-\gamma} 
        \right\}^{1/n}
        \\
       & =
        \left\{
            \rR_\theta \left( \frac{\lambda_n}{\lambda_0} \right)
        \right\}^{\theta/n}
        \left\{
            \rR_{1-\gamma} \left( \frac{C_n}{C_0} \right)
        \right\}^{(1-\gamma)/n}.
    \end{align*}
    %
    In view of Lemma~\ref{l:wpok}, the operator $\KK$ satisfies all the
    conditions of the local spectral radius result in Theorem~\ref{t:lsr2}.
    Hence, taking the limit and raising to the power of $1/\theta$, we have
    %
    \begin{equation*}
        r(\KK)^{1/\theta} = 
        \lim_{n\to \infty} \left\{
            \rR_\theta \left( \frac{\lambda_n}{\lambda_0} \right)
        \right\}^{1/n}
        \lim_{n\to \infty}
        \left\{
            \rR_{1-\gamma} \left( \frac{C_n}{C_0} \right)
        \right\}^{(1-1/\psi)/n}.
    \end{equation*}
    %
    Multiplying by $\beta$ and taking logs yields
    %
    \begin{equation*}
        \sS := \ln \beta + \frac{\ln r(\KK)}{\theta} 
        = \ln \beta 
            + \sS_\lambda 
            + \left(1-\frac{1}{\psi}\right) \sS_c,
    \end{equation*}
    %
    as was to be shown.
\end{proof}


\begin{proof}[Proof of Proposition~\ref{p:ar1}]
    In \eqref{eq:ml_ar1}, we prove only the expression for $\sS_\lambda$, 
    since the derivation of $\sS_c$ is similar (and easier).  

    By \eqref{eq:lambda_ar1} we have $\lambda_n / \lambda_0 =
    \exp\left(\sum_{t=1}^{n}h_{\lambda, t} \right)$, and hence
    %
    \begin{equation*}
        \sS_\lambda = \lim_{n \to \infty}
        \frac{1}{n}
        \ln 
        \rR_\theta 
        \left(
        \exp\left(\sum_{t=1}^{n}h_{\lambda, t} \right)
        \right).
    \end{equation*}
    %
    Note that
    %
    \begin{equation*}
        \sum_{t=1}^{n} h_{\lambda, t}
        \sim N(0, v_n)
        \;\; \text{ with } \;\;
        v_n := 
        \frac{s_\lambda^2}{(1-\rho_\lambda)^2}
        \left(
        n -
        \frac{2\rho_\lambda (1-\rho_\lambda^n)}{1-\rho_\lambda}
        +\frac{\rho_\lambda^2(1-\rho_\lambda^{2n})}{1-\rho_\lambda^2}
        \right).
    \end{equation*}
    %
    It follows that
    %
    \begin{equation*}
        \ln \rR_\theta 
        \left(\exp\left(\sum_{t=1}^{n}h_{\lambda, t}\right)\right) 
        = \frac{\theta v_n}{2}.
    \end{equation*}
    %
    Multiplying by $1/n$ and taking the limit in $n$ gives~\eqref{eq:ml_ar1}. 

    The expression \eqref{eq:iiss} for $\sS$ follows directly from
    Proposition~\ref{p:d}.
\end{proof}



\subsection{Remaining Proofs}

We now complete all remaining proofs.  As before, Assumption~\ref{a:q} is in
force.  $\cC$ is the set of continuous everywhere positive functions on $\XX$.
This corresponds to the interior of the positive cone of the Banach lattice $c
\XX$, endowed with the supremum norm.


\begin{lemma}
    \label{l:Ft}
    If $\sS < 0$, then 
    %
    \begin{equation}\label{eq:cpk}
        \lim_{t \downarrow 0} \frac{F(t)}{t} \, r(\KK) > 1
        \quad \text{and} \quad
        \lim_{t \uparrow \infty} \frac{F(t)}{t} \, r(\KK) < 1,
    \end{equation}
    %
\end{lemma}

\begin{proof}
    Let $\Lambda := \exp(\sS) = \beta \, r(\KK)^{1/\theta}$.  By hypothesis,
    $\Lambda < 1$.  Observe that
    %
    \begin{equation}
        \label{eq:rat}
        \frac{F(t)}{t}
        = \left\{ \frac{1 - \beta}{t^{1/\theta}} + \beta \right\}^\theta .
    \end{equation}
    %
    If, on one hand, $\theta < 0$, then $\Lambda < 1$ implies $\beta^\theta
    r(\KK) > 1$ and, in addition, \eqref{eq:rat}
    increases to $\beta^{\theta}$ as $t \to 0$.  Thus, the first inequality in
    \eqref{eq:cpk} holds.  The second inequality also holds because $F(t)/t
    \to 0$ as $t \to \infty$.  If, on the other hand, $\theta > 0$, then
    $\beta^\theta r(\KK)
    < 1$ and  \eqref{eq:rat} diverges to $+\infty$ as $t \to 0$, so the first
    inequality in \eqref{eq:cpk} holds.  The second inequality also holds
    because $F(t)/t \to \beta^\theta$ as $t \to \infty$.
\end{proof}


\begin{lemma}\label{l:llo}
    If $\sS < 0$, then $\TT$ is geometrically stable on $\cC$.  
\end{lemma}


\begin{proof}
    It suffices to show that, when $\sS < 0$, the operator $\TT$ satisfies 
    the conditions of Theorem~\ref{t:du} on $\cC$.  
    By (b)--(c) of Lemma~\ref{l:wpok}, $\TT$ is a self-map on $\cC$.
    In addition,
    $\TT$ is order preserving on $\cC$ because $\KK$ is order preserving and $F$
    is increasing, so $f, g \in \cC$ and $f \leq g$ implies $(F \circ \KK) f \leq
    (F \circ \KK) g$.  In other words, $\TT f \leq \TT g$.  Moreover, $\TT $ is either
    convex or concave because $\TT  = F \circ \KK$, the operator $\KK$ is linear,
    and $F$ is increasing and either concave or convex,
    depending on the value of $\theta$. 

    In the rest of the proof, we let
    $r := r(\KK)$ and let $e$ be the Perron--Frobenius eigenfunction.
    By (e) of Lemma~\ref{l:wpok}, $e \in \cC$ and $r > 0$.
    Since $e \in \cC$, the constants $\underline e := \min_{x
    \in \XX} e(x)$ and $\bar e := \max_{x \in \XX} e(x)$ are finite and strictly
    positive.  

    Fix $g_1, g_2$ in $\cC$. Let
    %
    \begin{equation*}
        \underline g := \min_{x \in \XX} \min\{g_1(x), g_2(x)\}
        \quad \text{and} \quad
        \bar g := \max_{x \in \XX} \max\{g_1(x), g_2(x)\}.
    \end{equation*}
    %
    Since $g_1, g_2 \in \cC$ and $\XX$ is compact, we have $\underline g > 0$
    and $\bar g < \infty$.

    To complete the proof, we need to show that
    (a)--(c) of Theorem~\ref{t:du} hold.  That is, we need to establish existence of
    $f_1, f_2 \in \cC$ such that 
    %
    \begin{equation*}
        f_1 \leq g_1, g_2 \leq f_2,
        \quad
        \TT f_1 \gg f_1
        \quad \text{and} \quad
        \TT f_2 \ll f_2.
    \end{equation*}
    %
    First we construct a function $f_1 \in \cC$ such that 
    $f_1 \leq g_1, g_2$ and  $\TT f_1 \gg f_1$.

    By $\sS < 0$ and Lemma~\ref{l:Ft}, there exists
    a $\delta_1 > 1$ and an $\epsilon > 0$ such that
    %
    \begin{equation}
        \frac{F(t)}{t} r \geq \delta_1
        \quad \text{whenever} \quad 0 < t < \epsilon.
    \end{equation}
    %
    Choose $c > 0$ such that $c \bar e \leq \underline g$ and $c r \bar e <
    \epsilon$.  Since  $c r e(x) < \epsilon$ for all $x \in \XX$, we have
    %
    \begin{equation}
        (\TT  ce)(x)
        = F(c \KK e (x))
        = F(c r e (x))
        = \frac{ F(c r e (x)) }{c r e (x)} c r e (x)
        \geq \delta_1 c e (x).
    \end{equation}
    %
    Hence, for $f_1 := c e$, we have $f_1 \leq c \bar e \leq \underline g \leq
    g_1, g_2$ and $\TT  f_1 \gg f_1$.

    It remains to construct a function $f_2 \in \cC$ such that 
    $g_1, g_2 \leq f_2$ and  $\TT f_2 \ll f_2$.

    Using Lemma~\ref{l:Ft} again, we can choose a $\delta_2 <1$ and
    finite constant $M$ such that
    %
    \begin{equation}
        \frac{F(t)}{t} r \leq \delta_2
        \quad \text{whenever} \quad t > M.
    \end{equation}
    %
    Let $a$ be a positive constant such that $a r \underline e > M$ and $a
    \underline e \geq \bar g$.  By the definition of $\underline e$ we have $a
    r e(x) > M$ for all $x \in \XX$, so
    %
    \begin{equation}
        (\TT  a e)(x)
        = F(a \KK e (x))
        = F(a r e (x))
        = \frac{ F(a r e (x)) }{a r e (x)} a r e (x)
        \leq \delta_2 a e (x).
    \end{equation}
    %
    Hence, for $f_2 := a e$, we have $f_2 \geq a \underline e \geq \bar g \geq
    g_1, g_2$ and $\TT  f_2 \ll f_2$.

    We conclude that $\TT $ satisfies the conditions of Theorem~\ref{t:du} on $\cC$.  
\end{proof}


\begin{lemma}
    \label{l:bkn}
    If $\TT $ has a nonnegative, nonzero fixed point in $L_1(\pi)$, then $\sS < 0$.
\end{lemma}

\begin{proof}
    This follows from Proposition~C2 from \cite{borovicka2020necessary}, which
    shows that $\exp(\sS) < 1$ whenever $\TT $ has a nonnegative, nonzero fixed
    point in $L_1(\pi)$.  (The operator $\TT $ is slightly different, due to
    absence of the preference shock, but the proof is otherwise identical.)  
\end{proof}

\begin{proof}[Proof of Theorem~\ref{t:bk}]
    Before proceeding, we note that the claim in part (b) of the theorem is
    equivalent to the statement that $\TT $ is geometrically stable on $\cC$.

    If (a) holds, so that that $\Lambda < 1$, then   $\TT $ is geometrically
    stable on $\cC$ by Lemma~\ref{l:llo}.  Conversely, if (a) fails, then $\TT $
    has no fixed point in $\cC$, by Lemma~\ref{l:bkn}.  The last claim in
    Theorem~\ref{t:bk} also follows from Lemma~\ref{l:bkn}.
\end{proof}

\begin{proof}[Proof of Proposition~\ref{p:wcr}]
    Let $\tau \colon \cC \to \cC$ be defined by $\tau g = (1-\beta)^{-1}
    g^{1/\theta}$.  Straightforward algebra shows that $\UU = \tau \TT
    \tau^{-1}$ on $\cC$.  Since $\theta \not= 0$ and $\beta \in (0,1)$,
    the map $\tau$ and its inverse $\tau^{-1} f = (1-\beta)^\theta f^\theta$
    are continuous on $\cC$ when $\cC$ is endowed with the supremum norm distance.  As
    a result, $(\cC, \TT)$ and $(\cC, \TT)$ are topologically conjugate
    dynamical systems, which
    in turn implies that $(\cC, \UU)$ is globally stable if and only if $(\cC,
    \TT)$ is globally stable.  The claim in Proposition~\ref{p:wcr} now
    follows from Theorem~\ref{t:bk}.
\end{proof}

\begin{proof}[Proof of Proposition~\ref{p:drt}]
    Similar to the proof of Lemma~\ref{l:llo}, we aim to apply
    Theorem~\ref{t:du}.

    First note that all statements of Lemma~\ref{l:wpok} also hold for the
    operator $\tilde \KK$. Therefore, $\tilde \TT $ is a self-map on $\cC$ and is
    order preserving. Since $t \mapsto \tilde F(t, x)$ is either convex or
    concave, $\tilde \TT $ is either convex or concave.

    The rest of the proof relies on a similar statement to Lemma~\ref{l:Ft}:
    if $\tilde \sS < 0$, then
    %
    \begin{equation}
        \label{eq:cpk2}
        \lim_{t \downarrow 0} \frac{\tilde F(t, x)}{t} \, r(\tilde \KK) > 1
        \quad \text{and} \quad
        \lim_{t \uparrow \infty} \frac{\tilde F(t, x)}{t} \, r(\tilde \KK) < 1
    \end{equation}
    %
    for all $x \in \XX$. To check that \eqref{eq:cpk2} holds, observe that
    %
    \begin{equation*}
        \frac{\tilde F(t, x)}{t} = \left\{
            \frac{1 - h(x)}{t^{1/\theta}} + h(x)\beta
        \right\}^\theta.
    \end{equation*}
    %
    If $\theta < 0$, $\tilde \sS < 0$ implies that $\bar a^\theta
    \beta^\theta r(\tilde \KK) > 1$. Then, we have
    %
    \begin{equation*}
        \lim_{t\to 0} \frac{\tilde F(t, x)}{t}r(\tilde \KK)
        = h(x)^\theta\beta^\theta r(\tilde \KK)
        \geq \bar a^\theta \beta^\theta r(\tilde \KK) > 1
    \end{equation*}
    %
    and $\lim_{t\to \infty} \tilde F(t, x)r(\tilde \KK)/t = 0 < 1$. On the
    other hand, if $\theta > 0$, $\tilde \sS < 0$ implies that $\bar a^\theta
    \beta^\theta r(\tilde \KK) < 1$. Then, we have
    %
    \begin{equation*}
        \lim_{t\to \infty} \frac{\tilde F(t, x)}{t}r(\tilde \KK)
        = h(x)^\theta\beta^\theta r(\tilde \KK)
        \leq \bar a^\theta \beta^\theta r(\tilde \KK) < 1
    \end{equation*}
    %
    and $\lim_{t\to 0} \tilde F(t, x)r(\tilde \KK)/t \to \infty > 1$.

    Given \eqref{eq:cpk2}, the conditions of Theorem~\ref{t:du} can be
    checked as in the proof of Lemma~\ref{l:llo}.
\end{proof}


\subsection{Computing the Stability Coefficient}\label{s:comp_ssy}


Regarding calculation of $\sS_c$ in the model of
\cite{schorfheide2018identifying}, we have
%
\begin{equation*}
    \rR_{1-\gamma} \left( \frac{C_T}{C_0} \right)
    = \left( 
            \EE \left( \frac{C_T}{C_0} \right)^{1-\gamma}
        \right)^{1/(1-\gamma)}
    = \left( 
        \EE \exp\left( (1-\gamma) \sum_{t=1}^T g_{c,t} \right)
        \right)^{1/(1-\gamma)}.
\end{equation*}
%
Using the definition of consumption growth from \eqref{eq:ssycg}, the
expectation term becomes
%
\begin{equation*}
    \EE 
    \exp
    \left( (1-\gamma) T \mu_c
         + (1-\gamma) \sum_{t=1}^T z_{t-1} 
         + (1-\gamma) \sum_{t=1}^T \sigma_{c, t-1} \, \xi_{c, t}
     \right)
\end{equation*}
%
and hence 
%
\begin{equation*}
    \left\{ \rR_{1-\gamma} \left( \frac{C_T}{C_0} \right) \right\}^{1/T}
    = \exp(\mu_c)
        \left[
            \rR_{1-\gamma} \exp \left( \sum_{t=1}^T z_{t-1} \right)
        \right]^{1/T}
        \left[
            \rR_{1-\gamma} \exp \left( \sum_{t=1}^T \sigma_{c, t-1} \, \xi_{c, t} \right)
        \right]^{1/T}.
\end{equation*}
%
Taking logs yields
%
\begin{equation*}
    \sS_c 
    = \mu_c + \sS_c^z + \sS_c^\sigma,
\end{equation*}
%
where
%
\begin{equation*}
    \sS_c^z 
    := \lim_{T \to \infty} \frac{1}{T} \ln 
        \rR_{1-\gamma} \exp \left( \sum_{t=1}^T z_{t-1} \right)
        = \lim_{T \to \infty} \frac{1}{T(1-\gamma)} \ln 
        \EE \exp \left( (1-\gamma) \sum_{t=1}^T z_{t-1} \right)
\end{equation*}
%
and
%
\begin{align*}
    \sS_c^\sigma
    & := \lim_{T \to \infty} \frac{1}{T} \ln 
    \rR_{1-\gamma} \exp \left( \sum_{t=1}^T \sigma_{c, t-1} \, \xi_{c, t} \right)
    \\
    & = \lim_{T \to \infty} \frac{1}{T(1-\gamma)} \ln 
    \EE \exp 
       \left( 
           \frac{(1-\gamma)^2}{2} \sum_{t=1}^T \sigma_{c, t-1}^2
       \right),
\end{align*}
%
where, in the second
equality, we used independence of $\{ \sigma_{c, t} \}$ and $\{
\xi_{c,t}\}$, as well as $\xi_{c,t} \sim N(0,1)$ for all $t$.

% bibliography


\subsection{Numerical Accuracy}
\label{ss:N_T}

In this section, we compute the stability coefficient $\sS$ using Monte Carlo
simulations for models that do not have an analytic expression for $\sS_c$.
Recall that $\sS_c$ is given by
%
\begin{equation*}
    \sS_c
    = \lim_{T \to \infty}
    \frac{1}{T} \ln 
    \rR_{1-\gamma} \left( \frac{C_T}{C_0} \right)
    = \lim_{T \to \infty} \frac{1}{T} \frac{1}{1-\gamma} \ln
        \EE \left( \frac{C_T}{C_0} \right)^{1-\gamma}
\end{equation*}
%
by Proposition~\ref{p:d}. To estimate it numerically, we generate $N$
independent consumption paths of length $T$ and evaluate
%
\begin{equation*}
    \hat{\sS}_c
    = \frac{1}{T} \frac{1}{1-\gamma} \ln \frac{1}{N}
    \sum_{n=1}^N \left( \frac{C^{(n)}_T}{C^{(n)}_0} \right)^{1-\gamma},
\end{equation*}
%
where $C_t^{(n)}$ is the $n$th simulation of time $t$ consumption. Here, we
use the sample average to estimate the expectation in $\sS_c$, the validity
of which is guaranteed by the Law of Large Numbers.

\begin{figure}
    \centering
    \includegraphics[width=0.6\textwidth]{figs/stab_coeff_N_T_ssy}
    \caption{The figure plots $\sS$ for different $N$ and $T$ for the model of \cite{schorfheide2018identifying}.}
    \label{f:ssy_NT}
\end{figure}


\begin{figure}
    \centering
    \includegraphics[width=0.6\textwidth]{figs/stab_coeff_N_T_gcy}
    \caption{The figure plots $\sS$ for different $N$ and $T$ for the model of \cite{GomezYaron2020}.}
    \label{f:gcy_NT}
\end{figure}

To evaluate how sample size and length affect the accuracy of the estimator
$\hat \sS_c$, we plot the estimated stability coefficient for different $N$
and $T$ for the model of \cite{schorfheide2018identifying} in
Figure~\ref{f:ssy_NT} and for the model of \cite{GomezYaron2020} in
Figure~\ref{f:gcy_NT}. The graphs show that for large $N$ and $T$, increasing
them further only has a marginal effect on the estimates. For the results
reported in the main text, we use $T = 100,000$ and $N =
10,000$.\footnote{\cite{borovicka2020necessary} use $T, N \leq 5,000$ in
  their applications, but we need much longer sample paths to accurately
  compute $\sS_c$ in the presence of preference shocks.}

\bibliographystyle{ecta}
\bibliography{localbib}

\end{document}


