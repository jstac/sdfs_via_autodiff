\documentclass[12pt, reqno]{amsart}

%%%%%%%%%%%%%%%%%%%%%  MY STUFF %%%%%%%%%%%%%%%%%%%%%%%%%%%%%%%%%%
%\makeatletter
%\g@addto@macro{\endabstract}{\@setabstract}
%\makeatother


%\usepackage{epsfig}
\usepackage{graphics, stackrel}
\usepackage{amsmath, amssymb, amsthm}
\usepackage{graphicx}
\usepackage{verbatim}
\usepackage{amsfonts}
\usepackage{natbib}
\usepackage{enumitem}
%font
%\usepackage{lmodern}
\usepackage[T1]{fontenc}
\usepackage{mathpazo}
%\usepackage{tgpagella}

%subfloats / figures
\usepackage{caption}
\usepackage{subcaption}

% For pandas latex tables
\usepackage{booktabs}


\usepackage{fancyvrb}
\usepackage[usenames,dvipsnames,svgnames,table]{xcolor}
\usepackage{mdwlist}

\usepackage[citecolor=blue, colorlinks=true, linkcolor=blue]{hyperref}


% lists
\usepackage{enumitem}
\setlist[enumerate]{itemsep=2pt,topsep=3pt}
\setlist[itemize]{itemsep=2pt,topsep=3pt}
\setlist[enumerate,1]{label=(\alph*)}

\usepackage{mathrsfs}  % caligraphic
%\usepackage{stix} 
\usepackage{bbm}
\usepackage{bm}        % bold symbols


%% page layout
\usepackage[left=1.25in, right=1.25in, top=1.0in, bottom=1.15in, includehead, includefoot]{geometry}

% nice inequalities
\renewcommand{\leq}{\leqslant}
\renewcommand{\geq}{\geqslant}

% inner product
\providecommand{\inner}[1]{\left\langle{#1}\right\rangle}
\providecommand{\innerp}[1]{\left\langle{#1}\right\rangle_\pi}


\usepackage[ruled, linesnumbered]{algorithm2e}

%extra spacing
\renewcommand{\baselinestretch}{1.25}

%horizonal line
\newcommand{\HRule}{\rule{\linewidth}{0.3mm}}

% skip a line between paragraphs, no indentation
\setlength{\parskip}{1.5ex plus0.5ex minus0.5ex}
\setlength{\parindent}{0pt}

% footnote without a maker (blfootnote)
\newcommand\blfootnote[1]{%
  \begingroup
  \renewcommand\thefootnote{}\footnote{#1}%
  \addtocounter{footnote}{-1}%
  \endgroup
}

\DeclareMathOperator{\Span}{span}
\DeclareMathOperator{\diag}{diag}
\DeclareMathOperator*{\argmin}{arg\,min}
\DeclareMathOperator*{\argmax}{arg\,max}

\DeclareMathOperator{\cl}{cl}
%\DeclareMathOperator{\overset{\circ}}{int}
\DeclareMathOperator{\Prob}{Prob}
\DeclareMathOperator{\determinant}{det}
\DeclareMathOperator{\Var}{Var}
\DeclareMathOperator{\Cov}{Cov}
\DeclareMathOperator{\graph}{graph}

% mics short cuts and symbols
\newcommand{\st}{\ensuremath{\ \mathrm{s.t.}\ }}
\newcommand{\setntn}[2]{ \{ #1 : #2 \} }
\newcommand{\fore}{\therefore \quad}
\newcommand{\preqsd}{\preceq_{sd} }
\newcommand{\toas}{\stackrel {\textrm{ \scriptsize{a.s.} }} {\to} }
\newcommand{\tod}{\stackrel { d } {\to} }
\newcommand{\tou}{\stackrel { u } {\to} }
\newcommand{\toweak}{\stackrel { w } {\to} }
\newcommand{\topr}{\stackrel { p } {\to} }
\newcommand{\disteq}{\stackrel { \mathscr D } {=} }
\newcommand{\eqdist}{\stackrel {\textrm{ \scriptsize{d} }} {=} }
\newcommand{\iidsim}{\stackrel {\textrm{ {\sc iid }}} {\sim} }
\newcommand{\1}{\mathbbm 1}
\newcommand{\la}{\langle}
\newcommand{\ra}{\rangle}
\newcommand{\dee}{\,{\rm d}}
\newcommand{\og}{{\mathbbm G}}
\newcommand{\ctimes}{\! \times \!}
\newcommand{\sint}{{\textstyle\int}}

\newcommand{\given}{\, | \,}
\newcommand{\A}{\forall}

% d for integrals
\newcommand*\diff{\mathop{}\!\mathrm{d}}
\newcommand*\e{\mathrm{e}}


% Special symbols and shortcuts
\newcommand{\bmeta}{\bm{\eta}}
\newcommand{\bmxi}{\bm{\xi}}

\newcommand{\infot}{\fF_t}

\newcommand{\pspace}{\mathscr{P}(\mathsf{X})}
\newcommand{\cspace}{\mathscr{C}(\mathsf{X})}

%\renewcommand{\times}{\! \times \!}

\newcommand{\aA}{\mathscr A}
\newcommand{\cC}{\mathscr C}
\newcommand{\sS}{\mathcal S}
\newcommand{\bB}{\mathscr B}
\newcommand{\oO}{\mathcal O}
\newcommand{\gG}{\mathcal G}
\newcommand{\hH}{\mathcal H}
\newcommand{\kK}{\mathcal K}
\newcommand{\iI}{\mathcal I}
\newcommand{\eE}{\mathcal E}
\newcommand{\fF}{\mathscr F}
\newcommand{\qQ}{\mathcal Q}
\newcommand{\tT}{\mathcal T}
\newcommand{\xX}{\mathcal X}
\newcommand{\yY}{\mathcal Y}
\newcommand{\rR}{\mathcal R}
\newcommand{\zZ}{\mathcal Z}
\newcommand{\wW}{\mathcal W}
\newcommand{\uU}{\mathcal U}
\newcommand{\lL}{\mathcal L}

\newcommand{\mM}{\mathcal M}
\newcommand{\dD}{\mathcal D}
\newcommand{\pP}{\mathcal P}

\newcommand{\vV}{\mathcal V}

\newcommand{\Bsf}{\mathsf B}
\newcommand{\Hsf}{\mathsf H}
\newcommand{\Vsf}{\mathsf V}

\newcommand{\BB}{\mathbbm B}
\newcommand{\DD}{\mathbbm D}
\newcommand{\RR}{\mathbbm R}
\newcommand{\CC}{\mathbbm C}
\newcommand{\QQ}{\mathbbm Q}
\newcommand{\NN}{\mathbbm N}
\newcommand{\GG}{\mathbbm G}
\newcommand{\UU}{\mathbbm U}
\newcommand{\TT}{\mathbbm T}
\newcommand{\YY}{\mathbbm Y}
\newcommand{\ZZ}{\mathbbm Z}
\newcommand{\HH}{\mathbbm H}
\newcommand{\MM}{\mathbbm M}
\newcommand{\PP}{\mathbbm P}
\newcommand{\EE}{\mathbbm E}


\newcommand{\bH}{\mathbf H}
\newcommand{\bT}{\mathbf T}

\newcommand{\var}{\mathbbm V}

\newcommand{\XX}{\mathsf X}
\newcommand{\II}{\mathsf I}
\newcommand{\WW}{\mathsf W}

\renewcommand{\phi}{\varphi}
\renewcommand{\epsilon}{\varepsilon}

\newcommand{\bP}{\mathbf P}
\newcommand{\bQ}{\mathbf Q}
\newcommand{\bE}{\mathbf E}
\newcommand{\bM}{\mathbf M}
\newcommand{\bX}{\mathbf X}
\newcommand{\bY}{\mathbf Y}

\theoremstyle{plain}
\newtheorem{theorem}{Theorem}[section]
\newtheorem{corollary}[theorem]{Corollary}
\newtheorem{lemma}[theorem]{Lemma}
\newtheorem{proposition}[theorem]{Proposition}


\theoremstyle{definition}
\newtheorem{definition}{Definition}[section]
\newtheorem{axiom}{Axiom}[section]
\newtheorem{example}{Example}[section]
\newtheorem{remark}{Remark}[section]
\newtheorem{notation}{Notation}[section]
\newtheorem{assumption}{Assumption}[section]
\newtheorem{condition}{Condition}[section]


%\DeclareTextFontCommand{\emph}{\bfseries}

%%%%%%%%%%%%%%%%%% end my preamble %%%%%%%%%%%%%%%%%%%%%%%%%%%%%%%%%%%






\begin{document}


\title{}




\begin{center}
    \LARGE 
    Solving High-Dimensional Asset Pricing Models using Automatic
    Differentiation and Accelerated Linear Algebra
    \blfootnote{To be added}
 
    \vspace{1em}

    %\large
    %Chase Coleman\textsuperscript{a}, Pablo Levi\textsuperscript{b}, John
    %Stachurski\textsuperscript{c}, \\ Ole Wilms\textsuperscript{d} and Junnan Zhang\textsuperscript{e} \par \bigskip

    %\small
    %\textsuperscript{a} Research School of Economics, Australian National University \\ 
    %\textsuperscript{b} Other affiliations to be added  \\ \bigskip

    \normalsize
    \today
\end{center}


\begin{abstract}
    To be written
    \vspace{1em}

    \noindent
    \textit{JEL Classifications:} D81, G11 \\
    \textit{Keywords:} Asset pricing, wealth-consumption ratio, automatic differentiation
\end{abstract}





\maketitle


%section
\section{Introduction}

As with many sub-fields in economics and finance, researchers working on asset
pricing face the need to handle a growing list of state variables when trying
to match existing theory to the data.  For example, while the heavily cited
paper of \cite{bansal2004risks} used just two state variables to model the
wealth-consumption ratio (a stationary expected growth rate process and a
stochastic volatility term), subsequent work has added many features to the
consumption side of this model.  For example,
\cite{schorfheide2018identifying} added preference shocks and an additional
stochastic volatility term, leading to four state variables.
\cite{gomez2021important} added two more state variables to handle
expectations of inflation.  

While six state variables is a tiny amount in some fields (e.g., linear
programming problems), quantitative analysis for asset pricing faces several
related problems.  First, models like the ones listed above are highly
nonlinear and, moreover, these nonlinearities matter for endogenous quantities
of interest, such as the risk premium (see, e.g., \cite{pohl2018higher}).
Second, discount factors are very close to unity (e.g., 0.998 in
\cite{bansal2004risks}), implying slow convergence of global iterative
solution methods.  Third, the consumption problem is embedded in a larger
model with still more state variables.  Fourth, the model needs to be solved
at many different parameterizations in order to fit the data.

In this paper we show how to rapidly approximate global solutions to the kinds
of asset pricing models listed above using GPU computing based on Google JAX.

Short history of JAX, discussion of its capabilities.

Some explanation of what we achieve.

Review of related literature.


\section{Asset Pricing Background}

In discrete-time no-arbitrage environments, the equilibrium price process
$\{P_t\}_{t \geq 0}$ associated with a cash flow $\{G_t\}_{t \geq 1}$ obeys
the fundamental recursion
%
\begin{equation}
    \label{eq:pd}
    P_t
    = \EE_t \, M_{t+1} ( P_{t+1} + G_{t+1} )
    \quad \text{for all } t \geq 0,
\end{equation}
%
where $\{ M_t\}$ is the sequence of single period stochastic discount factors
(see, for example,~\cite{kreps1981arbitrage}, \cite{hansen_richard:1987}
or~\cite{duffie2001dynamic}).\footnote{Equation~\eqref{eq:pd} assumes an
    ex-dividend contract, so that purchasing a claim to the asset at time $t$
entitles the buyer to the cash flow starting from date $t+1$.}  Most of the
well-known ``puzzles'' in asset pricing theory relate to the difficulty of
matching \eqref{eq:pd} to the data across a diverse range of asset classes.
Attempts to resolve these puzzles typically involve relatively sophisticated
models for the stochastic discount factor (SDF) process $\{ M_t\}$.

Most modern asset pricing models use 

\textcolor{blue}{For now a place to put some notes about ideas.}

Background on Newton's algorithm in $\RR^N$.  Let $f$ be a smooth map from
$\RR^N$ to itself.  We want to find the $x \in \RR^N$ that solves $f(x)=x$.
Ordinary successive approximation uses
%
\begin{equation}
    x_{k+1} = f(x_k)
\end{equation}
%
Newton's method first sets $g(x) = f(x) -x$, so that we are seeking a root $x$
satisfying $g(x)=0$, and then iterates on
%
\begin{equation}
    x_{k+1} = g(x_k) + J(x_k)^{-1} g(x_k)
\end{equation}
%
where $J(x)$ is the Jacobian of $g$ at $x$.

In the EZ fixed point problem, it's not too hard to compute an expression for
the Jacobian of the fixed point operator, so why use autodiff?  This is a fair
question.  Here are some thoughts?
%
\begin{itemize}
    \item We can also differentiate the fixed point with respect to the
        parameters. These gradients help us how the SDF and WC ratio respond
        to shifts in underlying parameters.  They can also provide Jacobians
        for gradient decent.
    \item It's possible to plug other fixed point operators, associated with
        other specifications of recursive utility, directly into the code.
        There's no need to compute gradients in each case.
    \item We might be able to use second order Newton methods, since autodiff
        can provide derivatives of any order.
\end{itemize}

General points:
%
\begin{itemize}
    \item By switching from successive approximation to Newton's method, using
        autodifferentiation to compute the Jacobian, we change the problem
        from many small iterations to a small number of computationally
        expensive ones, which offer better opportunities for parallelization. 
    \item Autodifferentiation is important here, as compared to numerical
        derivatives, since convergence is fragile for this operator (very slow
        rate of convergence).
\end{itemize}


\section{A Long Run Risk Model}

Roadmap to be added.

Consumption growth and the growth rate of the preference shock are given by
the generic formulas
%
\begin{equation}
    \label{eq:kappa}
    g_{c, t+1}
    = g_c(X_t, X_{t+1}, \xi_{t+1})
    \quad \text{and} \quad
    g_{\lambda, t+1}
    = g_\lambda(X_t, X_{t+1}, \xi_{t+1}),
\end{equation}
%
where $\{ X_t \}_{t \geq 0}$ is a discrete time Markov process on $\XX \subset \RR^d$, $\{
\xi_t \}_{t \geq 1}$ is an {\sc iid} process supported on $\YY \subset \RR^k$,
and $g_i \colon \XX \times \XX \times \YY \to \RR$ is continuous for
each $i \in \{c, \lambda\}$.  The processes $\{X_t\}$  and $\{\xi_t\}$ are
assumed to be independent.




In the model, the wealth-consumption ratio obeys
%
\begin{equation*}
    \beta^{\theta}
    \EE_t
    \left[
    \left( \frac{\lambda_{t+1}}{\lambda_t} \right)^\theta
        \left( \frac{C_{t+1}}{C_t} \right)^{1-\gamma}
        \left( \frac{w(X_{t+1})}{w(X_t)-1} \right)^\theta
    \right] = 1
\end{equation*}
%
Rearranging the previous expression gives
%
\begin{equation*}
    (w(X_t)-1)^\theta
    = \beta^{\theta}
    \EE_t
    \left[
    \left( \frac{\lambda_{t+1}}{\lambda_t} \right)^\theta
        \left( \frac{C_{t+1}}{C_t} \right)^{1-\gamma}
        w(X_{t+1})^\theta
    \right].
\end{equation*}
%

Let $\HH$ be the linear operator defined by
%
\begin{equation}\label{eq:defk}
    (\HH g)(x) = \EE_x 
        \, g(X_{t+1})  \,
        \exp
        \left[ 
            \theta g_\lambda(X_t, X_{t+1}, \xi_{t+1}) + (1-\gamma) g_c(X_t, X_{t+1}, \xi_{t+1}))
        \right]
\end{equation}
%
at each $x \in \XX$,  where $\EE_x$ conditions on $X_t = x$.
Conditioning on $X_t = x$, writing pointwise on $\XX$ and using the definition
of $\HH$ yields $w = 1 + \beta \left( \HH w^\theta \right)^{1/\theta}$.  A
function $w$ solves this equation if and only if $w$ is a fixed point
of the operator $\TT$ defined by 
%
\begin{equation}\label{eq:wcop}
    (\TT w) = 1 + \beta \,  (\HH w^\theta)^{1/\theta}.
\end{equation}
%



\subsection{The SSY Case}

In the long run risk model of \cite{schorfheide2018identifying}, the state
process takes the form 
%
\begin{equation*}
    X_t := (h_{\lambda, t}, h_{c, t}, h_{z, t}, z_t)  
\end{equation*}
%
where, for $i \in \{z, c, \lambda\}$,
%
\begin{align*}
    h_{i, t+1}
        & = \rho_i \, h_{i,t} + s_{i} \, \eta_{i, t+1}
        \\
    \sigma_{i,t} 
        & = \phi_i \, \bar{\sigma} \, \exp(h_{i,t}),
        \\
    z_{t+1} 
        & = \rho \, z_t + (1 - \rho^2)^{1/2} \, 
                \sigma_{z, t} \, \epsilon_{t+1}
\end{align*}
%


Consumption growth is given by 
%
\begin{equation}\label{eq:ssygc}
    g_{c, t+1}
    = \ln \frac{C_{t+1}}{C_t} 
    = \mu_c + z_t + \sigma_{c, t} \, \xi_{c, t+1} .
\end{equation}
%
The preference shock $\lambda_t$ grows as
%
\begin{equation*}
    g_{\lambda, t+1}
    = \ln \frac{\lambda_{t+1}}{\lambda_t} 
    = h_{\lambda, t+1}.
\end{equation*}
%
The innovations 
%
\begin{equation*}
    \xi_{c, t}, \;\;  \epsilon_t, \text{ and } (\eta_{i, t})_{i \in \{z, c, \lambda\}}
\end{equation*}
%
are all independent and standard normal.  


\subsection{Discretization}

For computation, the state process is distcretized onto a state space $S =
\{x_1, \ldots, x_N\}$ of size $N \in \NN$ and the operator $\HH$ is
represented by a matrix $\bH$, with 
%
\begin{equation}\label{eq:defkb}
    \bH(n, n') = \sum_{n'=1}^N 
        \,
        \exp
        \left\{ 
            \theta g_\lambda(x_n, x_{n'}, \xi) 
            + (1-\gamma) g_c(x_n, x_{n'}, \xi)
        \right\}
    \bP(n, n'),
\end{equation}
%
where $\bP$ is an $N \times N$ matrix with  $\bP(n, n')$ representing the
probability that the discretized state process transitions from state $x_n$ to
state $x_{n'}$ in one unit of time.

The discretization of $\TT$ is written as $\bT$ and the problem is to find the
fixed point of 
%
\begin{equation}\label{eq:wcopd}
    (\bT w) = 1 + \beta \,  (\bH w^\theta)^{1/\theta}
\end{equation}
%
in the set of strictly positive vectors in $\RR^N$.





\bibliographystyle{ecta}

\bibliography{localbib}


\end{document}
